\documentclass[12pt]{report}
\usepackage{graphicx}
\graphicspath{{images/}}
\usepackage[a4paper, width=150mm, bottom=25mm, bindingoffset=6mm]{geometry}
\usepackage[super,sort&compress,comma]{natbib} 
\usepackage{sidecap}
\usepackage[toc,page]{appendix}
\usepackage{amsmath}
\usepackage{multirow}
\usepackage{tabularx}

\usepackage{xcolor}
\definecolor{codegreen}{rgb}{0,0.6,0}
\definecolor{codegray}{rgb}{0.5,0.5,0.5}
\definecolor{codepurple}{rgb}{0.58,0,0.82}
\definecolor{backcolour}{rgb}{0.95,0.95,0.92}

\usepackage{listings}
\lstset{
  language=python,
  basicstyle=\small,
  breaklines=true,
    backgroundcolor=\color{backcolour},   
    commentstyle=\color{codegreen},
    keywordstyle=\color{magenta},
    numberstyle=\tiny\color{codegray},
    stringstyle=\color{codepurple},
    basicstyle=\ttfamily\footnotesize,
    breakatwhitespace=false,         
    breaklines=true,                 
    captionpos=b,                    
    keepspaces=true,                 
    numbers=none,                    
    numbersep=5pt,                  
    showspaces=false,                
    showstringspaces=false,
    showtabs=false,                  
    tabsize=2
  }

\renewcommand{\tablename}{Tabla}

\usepackage[labelfont=bf]{caption}


%\usepackage{fancyhdr}
%\pagestyle{fancy}
%\fancyhead{}
%\fancyhead[RO, LE]{}
%\fancyfoot[LE,RO]{\thepage}
%\renewcommand{\chaptermark}[1]{\markboth{#1}{}}
%\fancyfoot[LO, CE]{\leftmark}
%\fancyfoot[CO, RE]{Tomás Di Napoli}

\usepackage{fancyhdr}

% Define the page style for pages before the index
\fancypagestyle{beforeIndex}{%
    \fancyhf{} % Clear headers and footers
    \fancyfoot[C]{\thepage} % Roman page numbering centered
    \renewcommand{\thepage}{\roman{\thepage}} % Use Roman numerals for page numbering
}

\renewcommand{\chaptermark}[1]{%
    \markboth{Cap. \thechapter\ #1}{}
}

% Define the page style for pages after the index
\fancypagestyle{afterIndex}{%
    \fancyhf{} % Clear headers and footers
    % Header configuration
    \fancyhead[LO, LE]{\leftmark}
    \fancyhead[RO, RE]{Tomás Di Napoli}
    % Footer configuration
    \fancyfoot[LE]{\thepage} % Page number on the right for even pages
    \fancyfoot[RO]{\thepage} % Page number on the left for odd pages
    \fancyfoot[LO]{Tésis de licenciatura: Caracterización de UCNPs} % Chapter name on the left for odd pages
    \fancyfoot[RE]{Tésis de licenciatura: Caracterización de UCNPs} % Chapter name on the right for even pages
}





\pagestyle{afterIndex} % Normal numbering with alternating headers/footers



\usepackage{hyperref}


\usepackage[utf8]{inputenc}
\usepackage[spanish,activeacute]{babel}
\usepackage{xcolor}

\newcommand{\todo}[1]{\textcolor{red}{#1}}
\newcommand{\done}[1]{\textcolor{blue}{#1}}

\title{Tesis de lic}
\author{Tomás Di Napoli}

\begin{document}

%\maketitle

\thispagestyle{empty}
\begin{center}

%\vspace*{2cm}
{\includegraphics[width=5cm]{UBA_logo.png}\\}
\vspace*{2mm}
{\large Universidad de Buenos Aires\\}
{\large Facultad de Ciencias Exactas y Naturales\\}
{\large Departamento de Física\\}
%\vspace*{5mm}
%{ Intendente Güiraldes 2160, Buenos Aires, Argentina\\}
%\vspace*{5mm}


\vspace*{2cm}
{\huge \bf Renovación de un Espectrofluorímetro Obsoleto para la Caracterización de Nanopartículas de \textit{Upconversion}}

%\bf
% \vspace*{2.2cm}
% {\large A thesis submitted in partial fulfillment\\}
% {\large  of the requirements for the degree of \\}

\vspace*{1cm}
{\it {\large Tesis de Licenciatura en Ciencias Físicas} 
}

\vspace*{0.8cm}


\vspace*{6mm}
{\Large Tomás Di Napoli \\}
%{\large LU: 323/18\\
%{\small \tt azulmariabriganet@gmail.com}}

\vspace*{10mm}
{\large Dirección: Hernán Grecco\\}
\vspace*{5mm}
%{\large Colaboradores: Juan Cruz Berón\\}

\vspace*{5mm}
{\large Diciembre 2024\\}
\end{center}

\newpage

{\large TEMA: Renovación de un Espectrofluorímetro Obsoleto para la Caracterización de Nanopartículas de \textit{Upconversion}}\newline
%\vspace*{0.01cm}\newline

{\large ALUMNO: Tomás Di Napoli}\newline
%\vspace*{0.02cm}\newline

{\large LU: 551/18}
%\vspace*{0.02cm}\newline

{\large LUGAR DE TRABAJO: Laboratorio de Electrónica Cuántica, Departamento de Física, Instituto de Física de Buenos Aires, CONICET/UBA. }\newline
%\vspace*{0.02cm}\newline

{\large DIRECCIÓN: Hernán Grecco (IFIBA-UBA/CONICET)}\newline

%\vspace*{0.02cm}\newline
{\large FECHA DE INICIACIÓN: Agosto 2023}\newline
%\vspace*{0.02cm}\newline

{\large FECHA DE FINALIZACIÓN: Diciembre 2024}\newline
%\vspace*{0.02cm}\newline

{\large FECHA DE EXAMEN: 27/12/2024 }\newline
%\vspace*{0.02cm}\newline

{\large INFORME APROBADO POR: }

\vspace*{1.5cm}
\begin{center}
    \renewcommand{\arraystretch}{4.5} % Ajusta la altura de las filas
    \large % Tamaño de fuente más grande para el contenido de la tabla
    \begin{tabular}{|p{7cm}|p{7cm}|}
        \hline
        \multirow{2}{7cm}[4em]{ Autor/a \\ Tomás Di Napoli} & \multirow{2}{5cm}[4em]{ Jurado \\ Dr. Christian T. Schmiegelow} \\
        \hline
         \multirow{2}{7cm}[4em]{ Director/a \\ Dr. Hernán Grecco } & \multirow{2}{5cm}[4em]{ Jurado \\ Dra. María Gabriela Capeluto} \\
        \hline
        \multirow{2}{7cm}[4em]{ Profesor/a de la Tesis de Licenciatura\\Dra. Silvina Ponce Dawson} & \multirow{2}{5cm}[4em]{ Jurado\\ }\\
        \hline
    \end{tabular}
\end{center}

\pagenumbering{roman}

\chapter*{Agradecimientos}

\chapter*{Resumen}

La fotoluminiscencia es la emisión de luz de la materia debida a una absorción previa de fotones. 
Esta propiedad es frecuentemente utilizada en microscopía con aplicaciones bioquímicas y biológicas, generalmente a través de agentes fotoluminiscentes como tintas orgánicas, proteínas fluorescentes y puntos cuánticos semiconductores. 
La mayoría de estos agentes exhiben corrimiento de Stokes, conversión de fotones de alta energía (UV-vis) a baja energía (rojo-NIR).
En este trabajo se estudian las nanopartículas de \textit{upconversion} (UCNP), redes cristalinas dopadas con lantánidos que interactúan entre sí para absorber fotones en el NIR y re-emitirlos en el UV-VIS.

En particular, se desarrolló tanto el \textit{hardware} como el \textit{software} de una plataforma capaz de la caracterización óptica de estas nanopartículas, tomando como base un espectrofluorímetro comercial obsoleto Horiba PTI Quanta Master 400.
Se renovó el instrumento reemplazando la PC y electrónica de control por una CPU y FPGA Red Pitaya.
Se desarrollaron dos paquetes de Python, \textit{redpipy} y \textit{refurbishedPTI}, que perimten controlar los monocromadores y leer la señal del PMT del fluorímetro.
Ambos paquetes se desarrollaron de forma tal que puedan ser reutilizados: \textit{redpipy} consiste en una interfaz de aplicación del programador (API) para controlar la Red Pitaya, y \textit{refurbishedPTI}, que implementa el control del fluorímetro, no es particular de este instrumento, sino que se puede utilizar para controlar cualquier fluorímetro con monocromadores y PMT.
Para asegurarse de que el fluorímetro renovado implemente las funcionalidades correctamente se caracterizó en profundidad el sistema de detección de fotones.
Además, se expandieron las capacidades del fluorímetro original agregando un láser externo pulsado, con una fuente de potencia variable, que permite realizar mediciones de tiempo de vida aplicando la técnica \textit{Time-Correlated Single Photon Counting} (TCSPC).

Por último, se utilizó esta plataforma para caracterizar un lote de UCNP compuestas por una matriz cristalina de fluoruro de ítrio NaYF$_4$ y dopada con iones lantánidos erbio Er$^{3+}$ e iterbio Yb$^{3+}$.
Para eso, se midió el espectro de emisión estacionario al excitar de forma continua con un láser de 976 nm en un rango de densidades de potencia de excitación desde 16 mW cm$^{-1}$ hasta 80 mW cm$^-1$.
Asimismo, se midieron los tiempos de vida para los picos de emisión más intensos, y se comparó el resultado a dos potencias de excitación distintas.
Se observó un comportamiento lineal en función de la potencia para los espectros estacionarios, y no se observaron diferencias significativas entre los tiempos de vida a distintas potencias.

\textbf{Palabras clave:} \textit{upconversion}, luminiscencia, espectrofluorimetría, instrumentación y control.






\tableofcontents


\chapter{Introducción}
\pagenumbering{arabic}
\section{Espectrofotometría estática}
\section{Espectrofotometría dinámica}
\section{Nanopartículas de conversión ascendente}

\chapter{Espectrofluorímetro para la caracterización de UCNPs}
\section{Espectrofluorímetro HoribaPTI QuantaMaster 400}
\section{Renovación de HoribaPTI QuantaMaster 400}
\subsection{Hardware}
\subsection{Software}
\section{Expansión de Horiba PTI QuantaMaster 400 - Medición de tiempos de vida}
\subsection{Hardware}
\subsection{Software}


\chapter{Software y protocolos de medición}

\renewcommand{\tablename}{\textbf{Tabla}}

\section{Software} \label{sec:software}

\begin{figure}[t]
     \centering
     \includegraphics[width=0.8\textwidth]{software-diagram.png}
     \caption{\textbf{Estructura del \textit{software}}. Cada elemento del \textit{software} está ordenado de alto nivel (arriba) a bajo nivel (abajo). En amarillo se ven las componentes de \textit{refurbishedPTI}, en rojo las de \textit{redpipy}, en rosa la API y el \textit{hardware} de la RP.}
     \label{fig:code}
\end{figure}

\begin{figure}
    \centering
    \includegraphics[width=\textwidth]{imagen_soft.png}
    \caption{\textbf{Comparación del software antiguo y el nuevo.} (A) Imagen del \textit{software} original que controla al fluorímetro. (B) Imagen del \textit{software} de versiones más modernas del fluorímetro a las que se podría aplicar la renovación. (C) Imagen del \textit{software} renovado.}
    \label{fig:imagen_soft}
\end{figure}

Para reemplazar el rol que cumplía el \textit{software} FelixGX en el espectrofluorímetro original desarrollamos dos paquetes de Python de control de instrumental y adquisición de datos \cite{napoli_tdinapoli_2024,grecco_hgrecco_2024}.
El código corre en la CPU de la RP y permite controlar al espectrofluorímetro a través de una interfaz de programación de aplicaciones (API) y una interfaz gráfica simple (GUI) desarrollada con el paquete \textit{Jupyter Widgets} de \textit{IPython}.
El programa conformado por ambos paquetes está compuesto de cuatro capas principales (\textbf{Fig. \ref{fig:code}}):

\begin{itemize}
     \item \textbf{RedpiPy}: Es uno de los dos paquetes que desarrollamos. Consiste en un \textit{wrapper} (llamado \textit{rpwrap}) de la API original de la RP que resulta en que el código esté mejor organizado para hacer una aplicación en Python. Se compone de funciones y clases que permiten manejar el \textit{hardware} de la RP a bajo nivel, como \textit{RPDO} que controla los pines digitales, así como algunas clases de más alto nivel como \textit{Oscilloscope} que permite manejar el osciloscopio.
     \item \textbf{Clases de dispositivos}: Controlan componentes individuales del espectrofluorímetro, como los monocromadores, el láser pulsado, y los motores de los monocromadores permitiendo el control de todas las partes por separado. 
     \item \textbf{Clase Spectrometer}: Coordina el \textit{hardware} para protocolos de medición específicos (por ejemplo, adquirir un espectro de emisión). Es fácil de usar desde un script en Python o desde la línea de comandos. Además, es la encargada de contar los pulsos de voltaje negativo registrados por el PMT (ver sección \ref{sec:conteo}). Es utilizada por la interfaz gráfica.
     \item \textbf{Interfaz gráfica (GUI)}: proporciona herramientas de adquisición similares a las de FelixGX. Se accede a través de la web y utiliza el paquete \textit{Jupyter Widgets} de \textit{IPython}. En la figura (\textbf{\ref{fig:imagen_soft}}) se puede ver una comparación de las GUI.
\end{itemize}

Gracias a este diseño la parte del código que implementa el control del instrumento es completamente general y está desacoplada del resto, por lo que debería funcionar para cualquier modelo de espectrofluorímetro cuyos monocromadores sean controlados por motores por paso, y la señal de luminiscencia se lea con un PMT.
Por otro lado, la API pública le permite al usuario avanzado crear sus propios protocolos de medición que se pueden ejecutar sin la supervición de un operario.
Tanto \textit{RedpiPy}\footnote{Enlace al repositorio: \href{https://github.com/hgrecco/redpipy/}{https://github.com/hgrecco/redpipy/}} como \textit{RefurbishedPTI}\footnote{Enlace al repositorio: \href{https://github.com/tdinapoli/refurbishedPTI}{https://github.com/tdinapoli/refurbishedPTI}} (el paquete que controla al espectrofluorímetro) se encuentran públicos en repositorios de \href{https://github.com}{GitHub} con sus respectivas instrucciones de instalación. 
En el apéndice (\textbf{\ref{apendice:instalacion}}) se detallan los pasos para instalar el \textit{software} y en el apéndice (\textbf{\ref{apendice:instrucciones_uso}}) se dan ejemplos de cómo usarlo.
A continuación describo brevemente sus capacidades principales.

Si bien el \textit{software} se puede aplicar a un fluorímetro genérico, algunas características pueden variar de instrumento a instrumento, por ejemplo el rango de longitudes de onda que abarca, el paso mínimo de los motores del monocromador, la longitud de onda para la cual se activa el fin de carrera, entre otros parámetros.
A su vez, otro usuario podría requerir otras conexiones a los pines de la Red Pitaya a las que usamos nosotros.
Para poder modificar estos parámetros sin tener que modificar el código implementamos la lectura de cinco archivos de configuración: \texttt{emission\_init.yaml}, \texttt{emission\_calibration.yaml} y los análogos pero con \texttt{excitation}, e \texttt{itc\_config.yaml}.

Los archivos \texttt{\_calibration.yaml} contienen información que relaciona los parámetros de giro del motor, con los del monocromador. 
Por ejemplo, define las longitudes de onda máximas, y si la longitud de onda crece o decrece al girar en sentido horario (ver detalle en \ref{apendice:instalacion}).
Como estos parámetros son difíciles de conocer antes de utilizar los motores, desarrollamos un intérprete de línea, que está incluido en \textit{refurbishedPTI}, que controla solamente los motores y guía al usuario para que pueda completar los valores de los parámetros del archivo de calibración.
Por otro lado, los archivos \texttt{\_init.yaml} indican la relación entre la RP y los motores.
Permiten definir qué pines controlan a cada motor y sus fines de carrera respectivamente.
El archivo \texttt{itc\_config.yaml} configura los parámetros de inicialización de la fuente del láser externo.
Estos parámetros se pueden cambiar al utilizar el \textit{software}.

Por último, hay otros parámetros como el tiempo de integración máximo y mínimo, la ruta de destino donde se guardan los archivos por defecto, y el voltaje límite para el cual se binariza la señal (ver \ref{sec:caracterizacion_pmt}) que si bien no se leen desde archivos de calibración, se pueden modificar fácilmente desde el código.
El archivo \texttt{configs.py} contiene todos estos parámetros que luego se usan en el resto de archivos de \textit{refurbishedPTI}.

Para finalizar esta sección, voy a comentar brevemente el formato en el que se obtienen las mediciones.
Luego de una medición estacionaria se obtiene una tabla de datos en el formato \textit{DataFrame}, de la librería Pandas de Python.
Esta tabla contiene las columnas (i) longitud de onda, (ii) cuentas por segundo y (iii) tiempo de integración.
Además, el \textit{DataFrame} contiene metadatos de la medición, como el tipo de espectro que se tomó (excitación o emisión) y la longitud de onda del monocromador estático, la fecha y el tiempo de la medición, entre otros.
Las mediciones dinámicas también se guardan en un \textit{DataFrame} que contiene metadatos de la medición.


En las próximas secciones explicaremos cuál es el protocolo del software para hacer mediciones espectrales estacionarias y dinámicas, independientemente de la interfaz que se use para obtenerlas.




\section{Protocolo de medición estacionaria}

El espectro estacionario de emisión de una muestra consiste en la medición de su intensidad de luminiscencia al iluminar en una longitud de onda fija, y observar barriendo un rango de longitudes de onda.
La medición del espectro de excitación es análoga, pero cambia el rol de los monocromadores: en vez de observar la emisión en un rango de longitudes de onda, se observa en una longitud de onda fija y se barre un rango de longitudes de iluminación.
Por lo tanto, en el espectro de emisión, el monocromador estático es el de exctiación, y viceversa.
A continuación se detalla el protocolo de medición para un espectro de emisión estacionario.

Antes de iniciar una medición deben estar definidos sus parámetros que en este caso son el tiempo de integración $t_{int}$, la longitud de onda de iluminación $\lambda_e$ (la $e$ es por estático), y la longitud de onda inicial $\lambda_i$ y final $\lambda_f$ del barrido, así como el paso $\lambda_s$ entre cada medición de intensidad.
En el caso de tomar un espectro de UCNPs, dado que la iluminación proviene del diodo láser de 976 nm, también es necesario configurar la potencia óptica de excitación.

Una vez configurados los parámetros, el espectrofluorímetro debe realizar los siguientes pasos:

\begin{enumerate}
     \item \textbf{Inicializar los monocromadores} haciendo girar los motores en una misma dirección hasta que la señal del fin de carrera de cada uno sea de 5 V. Esto sirve para que la longitud de onda guardada por el \textit{software} coincida con la real.
     \item \textbf{Mover el monocromador estático} de emisión hasta $\lambda_e$. 
     \item \textbf{Mover el monocromador variable} de excitación hasta $\lambda_f$ en pasos de $\lambda_s$. Para cada longitud de onda los pasos (a) y (b) se deben repetir  $n$ veces, donde $n$ es tal que $n \times t_{max} \geq t_{int}$ y $t_{max}$ es el máximo tiempo de medición que soporta la RP (0.5 ms):
     \begin{enumerate}
          \item Medir la señal del PMT (\textbf{Fig. \ref{fig:diag_medicion_estatica}A}).
          \item Contar los picos en esa señal (ver sección \ref{sec:conteo}) y acumularlos. Al finalizar, el resultado es la cantidad de picos (fotones) contados por segundo.
     \end{enumerate}
\end{enumerate}

\noindent Una vez que el monocromador variable llega a $\lambda_f$ la cantidad de cuentas por segundo para cada longitud de onda se guarda en una tabla y termina la medición.
Al caracterizar UCNPs la excitación se da a través del láser externo, por lo que se deben configurar sus parámetros independientemente y el monocromador de excitación, que selecciona la longitud de onda de la lámpara que ilumina a la muestra, no toma ningún rol.
Como siempre se miden pantallas enteras, los tiempos de integración posibles son múltiplos de 0.5 ms.
Esto no es un problema porque los tiempos de integración necesarios suelen ser típicamente del orden de los segundos, dos o tres órdenes de magnitud mayores a la duración de la pantalla.
Además, la tabla de datos resultante de una medición contiene el tiempo de integración para cada punto con un error de $\sim 15$ ns.
En caso de que sea necesario medir con un tiempo de integración más preciso, esto se puede lograr modificando levemente el \textit{software} de \textit{refurbishedPTI}.

\begin{SCfigure}
     \centering
     \includegraphics[width=0.6\textwidth]{diag_medicion_estatica.png}
     \caption{\textbf{Diagrama de medición estacionaria.} En naranja se ve la señal del PMT. La línea punteada gris alta indica que el láser está en modo CW. En azul se ven las ventanas de la señal que lee la RP.}
     \label{fig:diag_medicion_estatica}
\end{SCfigure}


\section{Protocolo de medición dinámica} \label{sec:proceso_dinamico}

La medición de los tiempos de vida de las nanopartículas de \textit{upconversion} se realiza mediante la técnica de TCSPC (ver sección \ref{sec:intro_tcspc}).
Dado que estos tiempos de vida están en el rango de cientos de microsegundos, no son necesarios varios de los componentes de electrónica rápida típicos de la TCSPC utilizada en mediciones en el rango de nanosegundos, como el CFD y el TAC, los cuales son reemplazados por componentes más simples y menos costosos.
Contradictoriamente, esto hace que no se puedan caracterizar UCNPs utilizando equipos de fluorescencia de uso general, dado que el tiempo total de adquisición necesario difiere en órdenes de magnitud.
En nuestro caso, llevamos a cabo la técnica utilizando el trigger configurable a través de las entradas analógicas de la RP, y la señal TTL proveniente de la fuente de alimentación del láser.
Otra diferencia con TCSPC tradicional es el modo de excitación de la muestra.
Como la mayoría de fluoróforos orgánicos presentan su luminiscencia a través de la excitación de transiciones dipolares eléctricas, pérdia de energía por fonones, y re-emisión a través de otra transición dipolar, todos fenómenos que ocurren en el orden de los nanosegundos, es posible estudiar su espectro dinámico al excitar con un pulso del láser.
En el caso de las UCNPs, su luminiscencia se da por la dinámica no lineal de la interacción entre sus dopantes lantánidos (Yb$^{+3}$ y Er$^{+3}$), procesos que incluyen la excitación sucesiva sus electrones y por lo tanto ocurren en el orden de los microsegundos.
Por este motivo, es necesario iluminar a la muestra por algunos milisegundos para asegurarse de llegar al estado estacionario del sistema antes de medir su decaimiento.
Esto se hace aprovechando la función de alimentación pulsada (\textit{Quasi Continuous Wave} ó QCW) que ofrece la fuente ITC4020, la cual permite configurar frecuencia de pulsado $\nu$, y ciclo de trabajo $dc$ (\textbf{Fig. \ref{fig:diag_medicion_dinamica}}).

\begin{figure}
     \centering
     \includegraphics[width=\textwidth]{diag_medicion_dinamica.png}
     \caption{\textbf{Diagrama de medición dinámica.} La muestra es excitada intermitentemente con el láser (gris punteado). Al apagarse, el trigger de la RP se ejecuta y comienza a medir (azul) la señal (naranja) luego de esperar por $t_{ret}$. El resultado es un histograma (rojo) con la cantidad de fotones que llegaron en cada intervalo de tiempo.}
     \label{fig:diag_medicion_dinamica}
\end{figure}

Para hacer una medición de TCSPC es necesario definir la longitud de onda $\lambda$ en la que se detectará la emisión, el intervalo de tiempo $T$ en el que se van a contar los fotones luego del trigger, y la cantidad de veces $N$ que se va a medir ese intervalo.
Alternativamente, se podría definir un número de fotones $N_{fot}$ al que se quiere llegar, y medir el intervalo $T$ hasta que la cantidad de fotones medidos $n$ sea mayor a $N_{fot}$.
El protocolo por defecto de nuestro \textit{software} requiere determinar $N$.
Además, el intervalo de tiempo $T$ en el que se mide después del \textit{trigger} debe ser un múltiplo del tiempo máximo que puede medir la RP por ventana, $t_{max} = 0.5$ ms.
Por este motivo, en vez de especificar $T$, vamos a especificar $N_V$, el número de ventanas que queremos medir después del \textit{trigger} (\textbf{Fig. \ref{fig:diag_medicion_dinamica}}).
Entonces, el protocolo para realizar la medición es:

\begin{enumerate}
     \item \textbf{Inicializar los monocromadores} de forma análoga a la explicada en la sección anterior.
     \item \textbf{Mover el monocromador} de emisión hasta $\lambda$.
     \item \textbf{Iniciar el láser en modo QCW} y configurarlo para que se prenda y se apague con frecuencia $\nu$ y ciclo de trabajo $dc$.
     \item \textbf{Configurar la RP} para que espere un trigger en el canal analógico adecuado antes de medir.
     \item \textbf{Configurar un tiempo de retardo} $t_{ret} = (n_v - 1) \times t_{max}$, donde $1 \leq n_V \leq N_V$. $t_{ret}$ es el tiempo que la RP pasa sin medir después del \textit{trigger} para poder mover la ventana de medición (\textbf{Fig. \ref{fig:diag_medicion_dinamica}}). Repetir $N$ veces:
     \begin{enumerate}
          \item \textbf{Adquirir una ventana} después del \textit{trigger} y $t_{ret}$.
          \item \textbf{Encontrar los tiempos de llegada} de los pulsos usando el algoritmo explicado en la sección \ref{sec:conteo} y acumularlos.
     \end{enumerate}
     Al finalizar, el resultado es una tabla con los tiempos en los que llegaron los fotones después del \textit{trigger}.
\end{enumerate}

Con la tabla de datos final se construye un histograma (\textbf{Fig. \ref{fig:diag_medicion_dinamica}}), al cual se le puede ajustar un modelo de decaimiento para obtener el tiempo de vida de las partículas.


\section{Validación del \textit{software} y solución de problemas}

En esta sección comentaré brevemente algunos problemas que tuvimos a la hora de hacer el \textit{software} de control del instrumento y cómo los resolvimos.

Uno de los problemas que tuvimos fue que la adquisición de datos después del trigger no funcionaba correctamente cuando configurábamos retardos (ver sección anterior): los datos que se debían adquirir después del retardo se adquirían algunos microsegundos antes o después de lo esperado.
Este problema seguramente se daba por estar utilizando incorrectamente la API de la RP.
La solución fue crear \textit{redpipy} para tener una API más fácil de implementar sin cometer errores.
Para asegurarnos de que el problema del trigger ya no ocurría hicimos la siguiente prueba: medimos una señal de voltaje sierra 
%y (ii) utilizamos la RP como generador de funciones para simular un experimento de TCSPC.

Para esta prueba alimentamos a la RP con un generador de funciones externo emitiendo una señal tipo sierra a una frecuencia y voltaje pico a pico determinado.
Configuramos el trigger de la RP para que comenzara a medir en el momento en el que la señal pasa del voltaje máximo al mínimo.
Medimos la señal configurando retardos de distintos tiempos después del trigger, y vimos que utilizando \textit{redpipy} los datos se solapaban correctamente.

%Para la segunda prueba, generamos un vector numérico del mismo largo que la cantidad de muestas que puede medir la RP ($2^{14}$) con ceros y unos, donde cada número representaba un voltaje.
%Los unos están agrupados de a tres en el vector, de forma que representan pulsos cuadrados en la señal.
%Además, esos grupos están exponencialmente distribuidos a lo largo del vector. 
%La RP permite generar una señal eléctrica a partir de un vector, así que enviamos la señal correspondiente al vector por uno de los canales analógicos de la RP y la adquirimos en el otro.
%Como resultado

El segundo problema que tuvimos fue que la adquisición de un espectro estacionario llevaba mucho tiempo. 
Por ejemplo, medir la intensidad en una longitud de onda integrando por $0.1$ segundos llevaba 1 minuto.
Medir todo el espectro de esta forma resultaba inviable, ya que si se barren 300 nm con un paso de 1 nm integrando por $0.1$ segundos, la medición tardaría 5 horas.
Para entender cuál era el problema utilizamos el programa \textit{cProfile} que permite ejecutar un archivo de Python y devuelve la cantidad de tiempo que tardó en ejecutarse cada función que fue llamada.
Gracias a esto nos dimos cuenta de que una de las funciones que copia los datos de un lado a otro en la memoria de la RP estaba siendo llamada muchas veces y por lo tanto haciendo que el programa sea más lento.
Afortunadamente, encontramos otra función en la API de la RP que permite copiar un vector entero de datos.
Gracias a eso pudimos bajar el tiempo que lleva integrar $0.1$ segundos de 1 minuto a 10 segundos.
Este tiempo sigue siendo mayor que con el \textit{software} original, que era de $0.1$ segundos.
Por la falta de documentación no lo podemos saber con certeza, pero suponemos que esto es así porque el instrumento original utilizaba el PMT en modo analógico.
Si bien gracias a eso era más rápido para integrar, esto no le permitía medir tiempos de vida.

Por último, hicimos un programa que movía a los monocromadores entre distintas longitudes de onda y su longitud de onda inicial (la que activa el fin de carrera) para asegurarnos de que su control era correcto. 
Para eso, pusimos una cámara web apuntando al indicador físico de la longitud de onda del fluorímetro.
Luego, generamos números aleatorios dentro del rango posible de longitudes de onda, e hicimos que el monocromador recorra cada una de esas longitudes.
Al llegar a una longitud, el programa hacía una pequeña pausa para tomar una foto del indicador, y guardaba la longitud de onda a la que tenía que llegar en un archivo.
Al finalizar, verificamos que las longitudes de onda de las fotos y las guardadas en el archivo coincidían.









\chapter{Caracterización de UCNPs}


En este capítulo, vamos a utilizar el Horiba PTI QM 400 renovado y con su ampliación de capacidades para realizar mediciones de espectros estáticos de excitación y emisión, así como mediciones de tiempo de vida.
En la primera sección vamos a medir y comparar los espectros estacionarios de la rodamina B usando el instrumento renovado y el original.
En la segunda sección vamos a caracterizar ópticamente un lote de UCNPs.
Todas las partículas usadas en este trabajo fueron sintetizadas por el equipo colaborador del INQUIMAE, liderado por Beatriz Barja y María Claudia Marchi. 
Las nanopartículas utilizadas consisten en una red cristalina de fluoruro de ítrio NaYF$_4$ dopadas con los lantánidos iterbio (Yb) y erbio (Er), que en conjunto conforman una de las UCNPs más eficientes descriptas hasta el momento NaYF$_4$:Yb$^{+3}$,Er$^{+3}$ \cite{caracterizacion_ucnps_unicas}.
El método de síntensis empleado escapa el alcance de esta tésis, pero está detalladamente documentado en la literatura \cite{Zhang2012}.

\section{Espectros estacionarios de la rodamina B}

La rodamina es un fluoróforo extensamente usado en microscopía de fluorescencia debido a su fotoestabilidad y sus propiedades fotofísicas \cite{beija_synthesis_2009,rodamina_caracterizacion}.
Debido a su popularidad, sus espectros de excitación y emisión así como sus métodos de síntesis son ampliamente conocidos y reproducibles.
En particular, se caracteriza por tener un pico en 551 nm para la absorción y uno en 576 nm para la emisión.
A modo de verificación de que la medición de un espectro estático, tanto de excitación como de emisión, es el mismo utilizando el \textit{software} y el \textit{hardware} renovado y el original.
En la figura (\textbf{\ref{fig:rodamina}}) se ven los cuatro espectros normalizados por su máximo de intensidad.
Se puede ver que los espectros medidos con el instrumento original (azul) se solapan completamente con los medidos con nuestra renovación (naranja).
Dado que las transiciones electrónicas presentes en la rodamina son dipolares sus tiempos de vida medios son del orden de los nanosegundos, por lo que son imposibles de medir con nuestro instrumento con resolución mínima de $\sim 100$ ns.
En la siguiente sección utilizamos el espectrofluorímetro renovado para su propósito inicial: la caracterización óptica de UCNPs.

\begin{SCfigure}
    \centering
    \includegraphics[width=0.7\textwidth]{rho6b_validacion.png}
    \caption{\textbf{Espectros de la rodamina 6B}, tanto de excitación (punteado) como de emisión (sólido). Ambos fueron medidos con el \textit{software} y \textit{hardware} original (azul) como con el renovado (naranja).}
    \label{fig:rodamina}
\end{SCfigure}

\section{Caracterización óptica de UCNPs}

Nuestra plataforma adaptada nos permitió medir el espectro dinámico dependiente de la potencia de las UCNPs sintetizadas (\textbf{Fig. \ref{fig:power_dep_spectrum}}), y sus tiempos de vida (\textbf{Fig. \ref{fig:lifetimes}}) al ser excitadas con un láser de diodo IR de 976 nm.  
Como fue detallado en la sección \ref{sec:intro_ucnp}, la conversión ascendente es un proceso óptico no lineal en el que dos o más fotones se absorben secuencialmente entre niveles de energía igualmente espaciados, lo que lleva a la emisión de luz con una longitud de onda más corta que la incidente.  
Debido a las particularmente largas vidas medias (del orden de los cientos de microsegundos) y a los niveles de energía escalonados de los iones de tierras raras, se pueden observar espectros de conversión ascendente de UCNP en el rango visible incluso a bajas potencias de excitación.  
En este caso, medimos los espectros estáticos con densidades de potencia que varían entre 16 mW cm$^{-2}$ y 80 mW cm$^{-2}$ al excitar con el láser de forma continua (CW) (\textbf{Fig. \ref{fig:power_dep_spectrum}A}).  
El espectro estático muestra los picos de emisión conocidos de los iones Er$^{3+}$, que van desde la energía más baja hasta la más alta del espectro visible.  
Comenzando por el lado de menor energía, etiquetamos las regiones de emisión como rojo (R, 630–690 nm), amarillo (Y, 535–570 nm) y verde (G, 500–535 nm), correspondientes a las transiciones $^4$F$_{9/2} \to ^4$I$_{15/2}$, y $^4$S$_{3/2}, ^2$H$_{11/2} \to ^4$I$_{15/2}$, respectivamente.  
El espectro también muestra azul (B, 397–425 nm) y ultravioleta (UVA, 372–390 nm), que provienen de transiciones de niveles excitados más altos, como $^2$H$_{9/2} \to ^4$I$_{15/2}$ (410 nm) o $^4$G$_{11/2} \to ^4$I$_{15/2}$ (383 nm) \cite{haase_upconverting_2011}.  
Como se mencionó en la sección \ref{sec:intro_ucnp}, aunque las UCNPs muestran una relación no lineal entre la intensidad de emisión y la densidad de potencia de excitación, esto solo es apreciable cuando se abarcan varios órdenes de magnitud \cite{pollnau2000}.  
La intensidad de emisión de conversión ascendente ($I_{UC}$) está relacionada de manera no lineal con la densidad de potencia de excitación, $I_{UC} = P^\alpha$, donde $\alpha$ es el número efectivo de fotones involucrados en el proceso de absorción por fotón de mayor energía emitido, y $P$ es la potencia incidente \cite{Auzel2004}.  
El gráfico $\log{(I_{UC})}$ vs $\log{(P)}$ (\textbf{Fig. \ref{fig:power_dep_ranges}}) muestra que, para este rango de potencias, cada región de emisión está caracterizada por una pendiente relacionada con el número de fotones involucrados en cada proceso (\textbf{Fig. \ref{fig:power_dep_ranges} inset}).  
No se observaron cambios en estas pendientes para cada longitud de onda (UV, B, G, R), lo que indica que los mecanismos fotofísicos que causan la conversión ascendente son los mismos para todas las potencias.  
Esto está en línea con el hecho de que la relajación cruzada y la transferencia de energía hacia atrás (BET) desde Er$^{3+}$ hacia Yb$^{3+}$ son despreciables en estos límites de baja potencia \cite{suyver_anomalous_2005} \cite{berry_disputed_2015}.

\begin{figure}
    \centering
    \includegraphics[width=\textwidth]{spectrum_powerdens.png}
    \caption{\textbf{Espectro de las UCNPs dependiente de la potencia.} (A) Espectro de emisión bajo excitación CW por un diodo láser de 976 nm para distintas densidades de potencia de excitación entre 16 mW cm$^{-2}$ y 80 mW cm$^{-2}$.
    (B) Esquema de niveles de Yb$^{+3}$ y Er$^{+3}$, marcados con colores los niveles excitados del Er$^{+3}$ cuya emisión corresponde a las regiones del espectro.}
    \label{fig:power_dep_spectrum}
\end{figure}

Además de obtener los espectros estacionarios de las UCNPs, medimos el tiempo de vida medio en los picos de emisión de cada una de las regiones espectrales (\textbf{Fig. \ref{fig:lifetimes}}), esto es para las longitudes de onda 379, 410, 522, 541 y 654 nm.
Ajustamos los histogramas de tiempo de arribo de los fotones con una curva de decaimiento monoexponencial, comenzando a los 100 $\mu$s después de que se detuvo la excitación para medir solo la desocupación de los estados excitados.  
El resultado son vidas medias que varían entre $\sim125$ y $\sim400$ $\mu$s.  
Se observa una disminución en la vida media para los fotones de longitud de onda más corta, correspondientes a estados excitados de mayor energía, lo que es consistente con un mayor número de vías de relajación. 
Adicionalmente, medimos el tiempo de vida para el pico de emisión de 541 nm a dos potencias distintas de excitación, 3 y 87 mW \textbf{\ref{fig:lifetimes_power}}.
Al igual que con la relación entre potencia e intensidad, el cambio en tiempo de vida no presenta diferencias significativas para el rango de potencias en el que medimos.

Para este rango de potencias de excitación, no se llegan a medir diferencias significativas ni en la relación funcinoal entre potencia e intensidad, ni en el tiempo de vida para el pico más intenso de emisión.
En el futuro, se podrían repetir las mediciones excitando a densidades de potencia en el rango de 10$^0$ mW cm$^{-2}$ y 10$^2$ mW cm$^{-2}$, en el que se observan cambios en la pendiente de los gráficos $\log(I)$ vs $\log(P)$ \cite{bujjamer_luminescent_2020}.
Sin embargo, estas mediciones demuestran la posibilidad de realizar una caracterización óptica completa de las nanopartículas de \textit{upconversion} con las modificaciones que realizamos en el espectrofluorímetro QM 400.
Tanto las mediciones de espectro en función de la potencia como las de tiempo de vida fueron realizadas automáticamente con secuencias de comandos programadas en Python.
La plataforma permite desarrollar una secuencia de comandos más compleja que mida un espectro completo, detecte los picos de emisión y mida los tiempos de vida en esos picos, de forma que el usuario sólo tendría que colocar la muestra en la cámara y ejecutar el programa para realizar una caracterización óptica completa.

\begin{figure}
    \centering
    \includegraphics[width=\textwidth]{power_dep_ranges.png}
    \caption{\textbf{Gráfico Log-Log de I vs P.} Codificado con colores se encuentran las intensidades para cada potencia de excitación para cada uno de las regiones del espectro definidas. La tabla muestra la pendiente de un ajuste lineal para cada rango.}
    \label{fig:power_dep_ranges}
\end{figure}


\begin{figure}
    \centering
    \includegraphics[width=\textwidth]{lifetimes_residuos.png}
    \caption{\textbf{Histogramas de tiempos de vida.} Tiempo de arribo de los fotones (cuadrados) y ajuste de decaimiento exponencial (lineas sólidas) para cada región del espectro. Los tiempos de vida se midieron a 976 nm, 0.087 mW cm$^{-2}$.}
    \label{fig:lifetimes}
\end{figure}

\begin{SCfigure}
    \centering
    \includegraphics[width=0.7\textwidth]{lifetimes_power.png}
    \caption{\textbf{Tiempo de vida a distintas potencias.} Tiempo de arribo de los fotones (cuadrados) para el pico de emisión de 541 nm y dos potencias de excitación distintas, 3 mW cm$^{-2}$ y 87 mW cm$^{-2}$.}
    \label{fig:lifetimes_power}
\end{SCfigure}


\chapter{Conclusiones}
\renewcommand{\tablename}{\textbf{Tabla}}

Renovamos un espectrofluorímetro Horiba PTI Quanta Master 400 obsoleto.
Si bien sus elementos ópticos y de detección como los monocromadores y el PMT son componentes de altísima precisión, la utilización del instrumento resultaba dificultosa principalmente debido a su antigua PC de control con sistema operativo Windows 95.
Más allá de hacer tedioso su uso por la extracción de datos en disquette, la PC presenta una amenaza para la vida útil del fluorímetro: una falla en su funcionamiento puede dejarlo inutilizable y sin reparación, ya que el producto está fuera de soporte y sus partes no se consiguen en el mercado.
Nuestra renovación reemplazó la electrónica de control y la PC por una CPU y FPGA Red Pitaya, junto con dos paquetes de código abierto desarrollados en Python que cumplen el rol de \textit{software} de control y adquisición.
Para imitar el funcionamiento del fluorímetro original, hicimos una extensa caracterización del PMT y del algoritmo de conteo de fotones.

Además de la renovación, agregamos un láser externo pulsado de 980 nm que nos permitió adaptar el fluorímetro para hacer mediciones de espectros de emisión estáticos y dinámicos de nanopartículas de \textit{upconversion}.
Debido a los efectos no lineales que causan su luminiscencia, una caracterización óptica completa de estas partículas requiere medir sus espectros y tiempos de vida, del orden de milisegundos, para un rango de densidades de potencia de excitación.
Estas mediciones requieren el uso del fluorímetro por largos períodos de tiempo, en especial para potencias de excitación bajas en las que los tiempos de integración son largos.
Con nuestra plataforma estas mediciones se pueden programar en una rutina de Python, de forma tal que el usuario sólo debe depositar la muestra en la cubeta e iniciar la medición.

Por último, utilizamos nuestro desarrollo para hacer una caracterización óptica completa de UCNPs.
Las nanopartículas caracterizadas están compuestas por una matriz cristalina de fluoruro de ítrio dopada con iones lantánidos de iterbio y erbio, NaYF$_4$:Yb$^{+3}$,Er$^{+3}$, y fueron sintetizadas por el grupo de fotofísica del INQUIMAE, liderado por Beatriz Barja y María Claudia Marchi. 
Medimos su espectro estacionario y tiempos de vida para los picos de emisión en un rango de densidades de potencia entre 16 mW cm$^-1$ y 80 mW cm$^-1$.
Si bien los mecanismos no lineales de interacción entre los dopantes de las nanopartículas hacen que éstas tengan una respuesta no lineal con la potencia, no vemos ese comportamiento en nuestras mediciones.
Esto es porque el rango de potencias en el que excitamos es demasiado pequeño como para que haya un cambio en los mecanismos de \textit{upconversion}.

De cara al futuro, vemos que hay tres líneas en las que se puede progresar: (i) realizar mejoras en la plataforma que ya desarrollamos, (ii) aplicar nuestra plataforma a otros fluorímetros y (iii) aprovechar la automatización del instrumento para hacer mediciones más inteligentes.
En primer lugar, si bien la plataforma que desarrollamos cumple con la tarea de poder caracterizar UCNPs y renovar el fluorímetro, muchas de sus partes, como la GUI y la placa con la conexión a los instrumentos no son robustas.
Ahora que demostramos que las modificaciones funcionan, se podrían diseñar placas PCB, comprar conectores estandar y mejorar la GUI.

En segundo lugar, una vez que tengamos un diseño más robusto de las partes podríamos aprovechar para aplicar la renovación a los fluorímetros presentes en los institutos de Buenos Aires.

Por último, como se comentó en la introducción, la interacción entre los dopantes de las UCNPs se puede modelar con un sistema de ODEs con $\sim 50$ parámetros desconocidos.
Ajustar este modelo haciendo mediciones variando la potencia de excitación, longitud de onda de medición y ciclo de trabajo con fuerza bruta resulta impráctico.
En cambio, con nuestra plataforma se podría implementar un algoritmo que tome mediciones, haga un ajuste en tiempo real, y determine cuál es la próxima medición más informativa a realizar.
De esa forma, nuestra plataforma nos permitiría estudiar los mecanismos internos de interacción de las nanopartículas de \textit{upconversion}. \todo{quizás es muy vendehumo esto}





\bibliography{citas}
\bibliographystyle{rsc}

\begin{appendices}

\renewcommand{\thesection}{\Alph{section}}
\renewcommand{\thefigure}{A\arabic{figure}}
\renewcommand{\tablename}{Tabla}
\renewcommand{\thetable}{A\arabic{table}}

\section{Instrucciones de armado} \label{apendice:instrucciones_armado}

\subsection{Renovación del Horiba PTI Quanta Master 400} \label{subsec:refur-instructions}

El proceso de ensamblaje para la renovación del espectrómetro Horiba PTI Quanta Master 400 se puede organizar en cinco pasos: (1) conectar el motor de excitación $M_1$ (\textbf{Fig. \ref{fig:hardware}}), (2) conectar el fin de carrera de excitación, (3) conectar el motor de emisión $M_2$, (4) conectar el fin de carrera de excitación y (5) conectar la salida del PMT. 
Las fuentes de alimentación deben permanecer apagadas hasta que todo esté correctamente conectado, como se muestra en el esquema (\textbf{Fig. \ref{fig:schematic}}). 
Es necesario asegurarse de que el voltaje de GND sea el mismo en todas las conexiones. 
\textbf{Advertencia:} configurar límite de corriente del controlador DRV8825 esté configurado correctamente (en nuestro caso, para limitar a 0.7 A) y que los pines estén conectados adecuadamente, de lo contrario existe el riesgo de que circule demasiada corriente por los bobinados de los motores y se dañen.

\begin{enumerate}
    \item \textbf{Conectar el motor de excitación}
    \begin{enumerate}
        \item Conectar los pines P6 y P7 del RP a los pines STEP y DIR del DRV8825, respectivamente (\textbf{Fig. \ref{fig:schematic}}).
        \item Conectar los pines restantes: utilizando 3.3 V del RP como nivel alto digital, coloque los pines $\overline{\text{SLEEP}}$ y $\overline{\text{RESET}}$ del controlador del motor en alto, y conecte el GND lógico al GND del RP. Los pines $\overline{\text{ENABLE}}$, M0, M1, M2 y $\overline{\text{FAULT}}$ pueden dejarse flotantes.
        \item Ajustar el límite de corriente de salida del DRV8825 al máximo soportado por el motor paso a paso; en este caso, el motor M061CS02 tiene un límite de 0.7 A.
        \item Provea la fuente de alimentación del motor conectando los pines VMOT y GND a una fuente de 12 V que pueda suministrar al menos $2 \times \textit{límite de corriente}$. Conectar un capacitor de 100 $\mu F$ en paralelo.
        \item Conectar el controlador del motor al motor paso a paso: desconecte el conector original del motor $M_1$ y conecte los pines A1 y A2 del DRV8825 a los pines 1 y 7 del motor, y los pines B1 y B2 a los pines 3 y 5 (\textbf{Fig. \ref{fig:hardware}B}).
    \end{enumerate}
    \item \textbf{Conectar el fin de carrera de excitación}
    \begin{enumerate}
        \item Alimentar con 5 V y GND a los pines 1 y 2, respectivamente, del fin de carrera junto al motor $M_1$ (\textbf{Fig. \ref{fig:hardware}C}).
        \item Conectar el pin 3 del fin de carrera al pin digital P2 del RP para el fin de carrera de $M_1$, utilizando una resistencia pull-up externa al nivel lógico de 3.3 V del RP.
    \end{enumerate}
    \item \textbf{Conectar el motor de emisión}
    \begin{enumerate}
        \item Conectar los pines P4 y P5 del RP a los pines STEP y DIR del controlador del motor, respectivamente.
        \item Repetir los pasos (b) a (e) del ítem (1) para el motor $M_2$.
    \end{enumerate}
    \item \textbf{Conectar el fin de carrera de emisión}
    \begin{enumerate}
        \item Alimentar con 5 V y GND a los pines 1 y 2, respectivamente, del fin de carrera junto al motor $M_2$ (\textbf{Fig. \ref{fig:hardware}C}).
        \item Conectar el pin 3 del fin de carrera al pin digital P3 del RP para el fin de carrera de $M_2$, utilizando una resistencia pull-up externa al nivel lógico de 3.3 V del RP.
    \end{enumerate}
    \item \textbf{Conectar la salida del PMT}
    \begin{enumerate}
        \item Conectar el cable BNC-SMA a la entrada analógica 1 del RP y configurar el modo de alto voltaje.
        \item Conectar una ficha T BNC al extremo BNC del cable BNC-SMA conectado al RP.
        \item Conectar una terminación de 50 $\Omega$ a uno de los extremos de la ficha T para conformar pulsos de 40 a 100 ns.
        \item Conectar el otro extremo de la ficha T a la salida del PMT utilizando un cable BNC-BNC.
    \end{enumerate}
\end{enumerate}

\noindent Después de estos pasos, el instrumento estará listo para realizar mediciones estacionarias.

\begin{SCfigure}[][h]
         \centering
         \includegraphics[width=.65\textwidth]{schematic.png}
         \caption{\textbf{Esquemático de las conexiones} que contiene la placa PCB de prueba para conectar las componentes del hardware a la RP.
         }
         \label{fig:schematic}
    \end{SCfigure}


\subsection{Ampliación para mediciones dinámicas}

Para añadir la funcionalidad de medir tiempos de vida con el espectrómetro Horiba PTI, además de las instrucciones anteriores, deben realizarse los siguientes pasos adicionales:

\begin{enumerate}
    \item Conectar la fuente de luz externa al puerto externo utilizando un cable de fibra óptica.
    \item Conectar el puerto USB tipo B del ITC4020 al puerto USB tipo A del RP.
    \item Conectar la salida BNC R3 (salida TTL de 5 V del ITC4020) en el panel trasero del ITC4020 a un adaptador BNC a SMA, y luego conectarlo a la entrada analógica 2 del RP configurando ese canal en modo de alto voltaje.
\end{enumerate}

Nuestro software está diseñado para trabajar con el controlador láser ITC4020, pero se puede utilizar otro siempre y cuando se añadan clases de control específicas al software.

\subsection{Instalación y configuración del software} 

Seguir las instrucciones en \href{https://github.com/tdinapoli/refurbishedPTI}{este enlace}\cite{napoli_tdinapoli_2024} para instalar nuestro paquete Python en su RP. 
Una vez conectados los componentes de hardware, se deben calibrar los motores de los monocromadores y las entradas analógicas de la Red Pitaya. 
Las instrucciones para calibrar las entradas analógicas del RP están disponibles en su documentación oficial \cite{rp_docs}.

Para calibrar los monocromadores, desarrollamos un intérprete de línea de comandos que guía el proceso paso a paso y genera un archivo YAML con los parámetros de calibración (\textbf{Tabla \ref{tab:monochromator_api_parameters}}). El siguiente código en Python abre el menú para calibrar el monocromador de emisión:

\begin{lstlisting}[language=Python]
from refurbishedPTI import Spectrometer
spec = Spectrometer.constructor_default()
spec.emission_mono.calibrate()
\end{lstlisting}

El comando \texttt{save\_to\_yaml} guarda el archivo de configuración en el directorio del RP indicado por el método \texttt{.get\_config\_path()} de la clase Motor. Una vez calibrado el monocromador de emisión, repita el proceso para el monocromador de excitación:

\begin{lstlisting}[language=Python]
spec.excitation_mono.calibrate()
\end{lstlisting}

Repetir los mismos pasos que para el monocromador de emisión para completar la configuración.

\begin{table}[h]
 \centering
 \begin{tabular}{|l|l|p{10cm}|}
    \hline
    \textbf{Parámetro} & \textbf{Tipo de dato} & \textbf{Descripción} \\ \hline
    \texttt{greater\_wl\_cw}          & bool               & True si la longitud de onda incrementa al girar en sentido horario, Falso en el caso contrario. \\ \hline
    \texttt{max\_wl}                 & float              & Longitud de onda máxima (en nanómetros) que permitirá configurar la API del monocromador \\ \hline
    \texttt{min\_wl}                 & float              & Longitud de onda mínima (en nanómetros) que permitirá configurar la API del monocromador \\ \hline
    \texttt{wl\_step\_ratio}          & float              & Cambio (en nanómetros) de longitud de onda por cada paso que da el motor.\\ \hline
    \texttt{home\_wavelength}        & float              & Longitud de onda (en nanómetros) en la que se activa la señal del fin de carrera. \\ \hline
\end{tabular}
\caption{\textbf{Parámetros de la API de la clase Monochromator.}}
\label{tab:monochromator_api_parameters}
\end{table}

\section{Instrucciones de uso} \label{apendice:instrucciones_uso}

Como se comentó en la sección \ref{sec:software}, hay dos formas de operar el espectrómetro renovado: a través de la API de Python y mediante la interfaz gráfica Jupyter Notebook con IPython Widgets. 

Para ambos modos de operación, todos los instrumentos listados en la explicación de la sección \ref{subsec:refur-instructions} deben estar conectados, y tanto el PMT como la lámpara deben estar encendidos. 
Se deben ajustar las rendijas de Em y Ex (\textbf{Fig. \ref{fig:ref-diagram}}) según las necesidades del experimento. 
Una vez completados estos pasos, se puede proceder con las siguientes secciones para el modo de operación por script o GUI.

\subsection{Modo de operación GUI}

La interfaz gráfica del espectrómetro se encuentra en un Jupyter Notebook que permite al usuario cambiar los parámetros del instrumento mediante Widgets de Jupyter.  
La GUI se compone de dos secciones: el panel de parámetros y el panel de gráficos (\textbf{Figs. \ref{fig:spectrum_gui}} y \textbf{\ref{fig:lifetime_gui}}).  
El panel de parámetros contiene menús desplegables, botones y campos de texto para especificar los parámetros de medición y del archivo de medición.  
El panel de gráficos incluye dos gráficos, uno para mediciones de espectro y otro para mediciones de tiempo de vida.  

Para inicializar el modo GUI del QM400 renovado, se debe abrir el notebook \texttt{gui.ipynb} ubicado en \texttt{/home/jupyter/refurbishedPTI/gui.ipynb}.  
Al ejecutar la primera celda del notebook con el código:

\begin{lstlisting}[language=Python]
from refurbishedPTI.gui import Gui
gui = Gui()
\end{lstlisting}

\noindent se inicializa la GUI. 

Para realizar una medición seleccionar las opciones \textbf{Spectrum} o \textbf{Lifetime} en el menú desplegable \textbf{Measurement type}.  
Se especifican los parámetros de la medición utilizando los componentes de la GUI (detallados en las \textbf{Tablas \ref{tab:spectrum_measurement}} y \textbf{\ref{tab:lifetime_measurement}}) y comenzar la adquisición.  
Una vez finalizada la medición, guarde y manipule los datos con los componentes de la GUI (\textbf{Tabla \ref{tab:file_parameters}}).

\begin{table}
    \centering
    \begin{tabularx}{\textwidth}{|l|X|}
        \hline
        \textbf{Parámetro} & \textbf{Descripción} \\
        \hline
        \multirow{3}{3cm}{Spectrum type} & \textbf{Emission}: monocromador de excitación fijo y monocromador de emisión se mueve. \\
        \cline{2-2}
        & \textbf{Excitation}: monocromador de emisión fijo y monocromador de excitación se mueve. \\
        \cline{2-2}
        & \textbf{Laser}: monocromador de emisión se mueve y se usa el láser externo para excitar. \\
        \hline
        Static monochromator wavelength & Longitud de onda del monocromador fijo (nm). \\
        \hline
        Starting wavelength & Longitud de onda inicial del rango escaneado (nm). \\
        \hline
        Ending wavelength & Longitud de onda final del rango escaneado (nm). \\
        \hline
        Wavelength step & Diferencia en longitud de onda entre cada dato(nm). \\
        \hline
        Acquire & Comienza la medición. \\
        \hline
    \end{tabularx}
    \caption{\textbf{Parámetros de configuración de una medición de espectro estático}.}
    \label{tab:spectrum_measurement}
\end{table}

\begin{table}[htbp]
    \centering
    \begin{tabularx}{\textwidth}{|l|X|}
        \hline
        \textbf{Parámetro} & \textbf{Descripción} \\
        \hline
        Pump power & Potencia de excitación del láser (mW). \\
        \hline
        Frequency & Frecuencia de encendido y apagado del láser. \\
        \hline
        Duty Cycle & Ciclo de trabajo del láser en \%. \\
        \hline
        Emission monochromator wavelength & Longitud de onda para la cual se medirá el tiempo de vida (nm). \\
        \hline 
        Amount of counts & Cantidad de cuentas a medir antes de terminar la medición. \\
        \hline
        Starting time & Tiempo después del \textit{trigger} hasta empezar a contar (ms). \\
        \hline
        Ending time & Tiempo después del \textit{trigger} antes de parar de contar (ms). \\
        \hline
        Acquire & Comienza la adquisición.\\
        \hline
    \end{tabularx}
    \caption{\textbf{Parámetros de configuración de una medición dinámica}.}
    \label{tab:lifetime_measurement}
\end{table}

\begin{table}[htbp]
    \centering
    \begin{tabularx}{\textwidth}{|l|X|}
        \hline
        \textbf{Parámetro} & \textbf{Descripción} \\
        \hline
        Measurement filename & Seleccione un nombre de archivo y un directorio donde se guardará la medición una vez que se presione el botón \textbf{Save}. Si no se selecciona un nombre de archivo al momento de presionar el botón \textbf{Acquire}, el nombre de archivo será la fecha y hora actuales. \\
        \hline
        Selected measurement & Seleccionar una medición para guardarla o eliminarla. \\
        \hline
        Save & Guardar una medición con el nombre, directorio y formato seleccionado. \\
        \hline
        Delete & Eliminar la medición seleccionada. \\
        \hline
        \multirow{3}{3cm}{Save to} & Formato del arcihvo en el que se guardarán los datos. Opciones: \\
        \cline{2-2}
        & \textbf{pickle} \\
        \cline{2-2}
        & \textbf{csv} \\
        \cline{2-2}
        & \textbf{excel} \\
        \hline
    \end{tabularx}
    \caption{\textbf{Parámetros de configuración del archivo que guarda los datos de una medición}.}
    \label{tab:file_parameters}
\end{table}

\begin{figure}
    \centering
    \includegraphics[width=\textwidth]{spectrum_gui.png}
    \caption{\textbf{GUI para medir espectros estacionarios}.}
    \label{fig:spectrum_gui}
\end{figure}

\begin{figure}
    \centering
    \includegraphics[width=\textwidth]{lifetime_gui.png}
    \caption{\textbf{GUI para medir tiempos de vdia}.}
    \label{fig:lifetime_gui}
\end{figure}


\end{appendices}

\end{document}

