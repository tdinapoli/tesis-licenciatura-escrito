

\section{Software} \label{sec:software}

Para reemplazar el rol que cumplía el \textit{software} FelixGX en el espectrofluorímetro original desarrollamos dos paquetes de Python de control de instrumental y adquisición de datos \cite{napoli_tdinapoli_2024,grecco_hgrecco_2024}.
El código corre en la microCPU de la RP y permite controlar al espectrofluorímetro a través de una interfaz de programación de aplicaciones (API) y una interfaz gráfica simple (GUI) desarrollada con el paquete \textit{IPython's Jupyter Widgets}.
El programa conformado por ambos paquetes está compuesto de cuatro capas principales (\textbf{Fig. \ref{fig:code}}):

\begin{itemize}
     \item \textbf{RedpiPy}: Es uno de los dos paquetes que desarrollamos. Consiste en un \textit{wrapper} de la API original de la RP que resulta en que el código esté mejor organizado para hacer una aplicación en Python. Se compone de funciones y clases que permiten manejar el \textit{hardware} de la RP a bajo nivel, como \textit{RPDO} que controla los pines digitales, así como algunas clases de más alto nivel como \textit{Oscilloscope} que permite manejar el osciloscopio.
     \item \textbf{Clases de dispositivos}: Controlan componentes individuales del espectrofluorímetro, como los monocromadores, el láser pulsado, y los motores de los monocromadores permitiendo la el control de todas las partes por separado. 
     \item \textbf{Clase Spectrometer}: Coordina el \textit{hardware} para protocolos de medición específicos (por ejemplo, adquirir un espectro de emisión). Es fácil de usar desde un script en Python o desde la línea de comandos. Además, es la encargada de contar los pulsos de voltaje negativo registrados por el PMT. Es utilizada por la interfaz gráfica.
     \item \textbf{Interfaz gráfica (GUI)}: proporciona herramientas de adquisición similares a las de FelixGX. Se accede a través de la web y utiliza el paquete IPython's Jupyter Widgets.
\end{itemize}

Gracias a este diseño la parte del código que implementa el control del instrumento es completamente general y está desacoplada del resto, por lo que debería funcionar para cualquier modelo de espectrofluorímetro cuyos monocromadores sean controlados por motores por paso, y la señal de luminiscencia se lea con un PMT.
Por otro lado, la API pública le permite al usuario avanzado crear sus propios protocolos de medición que se pueden ejecutar sin la supervición de un operario.
Tanto \textit{RedpiPy} como \textit{RefurbishedPTI} (el paquete que controla al espectrofluorímetro) se encuentran públicos en repositorios de \href{https://github.com}{GitHub} con sus respectivas instrucciones de instalación. 
El apéndice \ref{apendice:instrucciones_uso} explica detalladamente algunos ejemplos que muestran cómo medir un espectro estacionario y el tiempo de vida para distintas longitudes de onda, ambos a través de la API y de la GUI.


%\todo{ \\
%capacidades del software \\
%\\
%Jerarquía de clases API para desacoplar. Extensión del software: sirve para espectrofluorímetro genérico siempre y cuando tengas motor por pasos y pmt.\\
%\\
%quien cuenta los picos, (mencionar que hay analisis de picos más adelante)\\
%\\
%Modos de uso, API GUI (capacidades, más detallado en apéndice).\\
%}

\begin{figure}[h]
     \centering
     \caption{\todo{cambiar esta figura} \textbf{Structure of the software}. Each element of the software is ordered from high level (\textbf{top}) to low level (\textbf{bottom}). Inside the dashed line black box In yellow, the two ways the end user can interact with the software. In orange, the refurbished instrument API classes. In red, the RP's hardware API.}
     \includegraphics[scale=0.3]{software-diagram.png}
     \label{fig:code}
\end{figure}

%\todo{introduccion a medicion de tiempos de vida}

El tiempo máximo de adquisición de 8 ms ofrece una ventana suficientemente amplia para la mayoría de los materiales luminiscentes. 
Como se mencionó anteriormente, cuando se requiere medir tiempos de vida mucho más largos al tiempo de adquisición máximo (0.5 ms), la ventana de detección puede desplazarse respecto al \textit{trigger} para cubrir un rango más amplio. 
El \textit{software} permite configurar la cantidad de veces que se mide cada ventana de 0.5 ms, lo que hace posible construir una curva de decaimiento más extensa de la luminiscencia.

%\todo{ \\
%explicar que es lo mismo que antes salvo que se usa el trigger.\\
%Explicar funcionamiento de las pantallas y offsets \\
%}

\section{Protocolo de medición estática}

El espectro estático de emisión(excitación) de una muestra consiste en la medición de su intensidad de luminiscencia al iluminar(observar) en una longitud de onda fija, y observar(iluminar) barriendo un rango de longitudes de onda.
Por lo tanto, antes de iniciar una medición deben estar definidos sus parámetros que en este caso son el tiempo de integración $t_{int}$, la longitud de onda de iluminación(medición) $\lambda_e$, y la longitud de onda inicial $\lambda_i$ y final $\lambda_f$ del barrido, así como el paso $\lambda_s$ entre cada medición de intensidad.
En el caso de tomar un espectro de UCNPs, dado que la iluminación proviene del diodo láser de 976 nm, también es necesario configurar la potencia óptica de excitación.

Una vez configurados los parámetros, el espectrofluorímetro debe realizar los siguientes pasos:

\begin{enumerate}
     \item \textbf{Inicializar los monocromadores} haciendo girar los motores en una misma dirección hasta que la señal del fin de carrera de cada uno sea de 5 V. Esto sirve para que la longitud de onda guardada por el \textit{software} coincida con la real.
     \item \textbf{Mover el monocromador estático} de emisión(excitación) hasta $\lambda_e$. 
     \item \textbf{Mover el monocromador dinámico} de excitación(emisión) hasta $\lambda_f$ en pasos de $\lambda_s$. Para cada longitud de onda los pasos (a) y (b) se deben repetir (\textbf{Fig. \ref{fig:diag_medicion_estatica}A}) $n$ veces, donde $n$ es tal que $n \times t_{max} \geq t_{int}$ y $t_{max}$ es el máximo tiempo de medición que soporta la RP (8 ms):
     \begin{enumerate}
          \item Medir la señal del PMT.
          \item Contar los picos en esa señal y acumularlos. Al finalizar, el resultado es la cantidad de picos (fotones) contados por segundo.
     \end{enumerate}
\end{enumerate}

\noindent Una vez que el monocromador dinámico llega a $\lambda_f$ la cantidad de cuentas por segundo para cada longitud de onda se guarda en una tabla y termina la medición.
Al caracterizar UCNPs la excitación se da a través del láser externo, por lo que se deben configurar sus parámetros independientemente y el monocromador de excitación no toma ningún rol.
Como siempre se miden pantallas enteras, los tiempos de integración posibles siempre son múltiplos de 8 ms.
Sin embargo, los tiempos de integración necesarios suelen ser típicamente del orden de los segundos, dos o tres órdenes de magnitud mayores a la duración de la pantalla.
Además, los datos también contienen el tiempo de integración para cada punto con un error de $\sim 15$ ns.
En caso de que sea necesario medir con un tiempo de integración más preciso, esto se puede lograr modificando levemente el \textit{software}.

\begin{SCfigure}
     \centering
     \includegraphics[width=0.6\textwidth]{diag_medicion_estatica.png}
     \caption{\textbf{Diagrama de medición estática}}
     \label{fig:diag_medicion_estatica}
\end{SCfigure}

\section{Protocolo de medición dinámica} \label{sec:proceso_dinamico}

La medición de los tiempos de vida de las nanopartículas de \textit{upconversion} se realiza mediante la técnica de TCSPC (ver sección \ref{sec:intro_tcspc}).
Dado que estos tiempos de vida están en el rango de cientos de microsegundos, no son necesarios varios de los componentes de electrónica rápida típicos de la TCSPC utilizada en mediciones en el rango de nanosegundos, como el CFD y el TAC, los cuales son reemplazados por componentes más simples y menos costosos.
Contradictoriamente, esto hace que no se puedan caracterizar UCNPs utilizando equipos de fluorescencia de uso general, dado que el tiempo total de adquisición necesario difiere en órdenes de magnitud.
En nuestro caso, llevamos a cabo la técnica utilizando el trigger configurable a través de las entradas analógicas de la RP, y la señal TTL proveniente de la fuente de alimentación del láser.
Otra diferencia con TCSPC tradicional es el modo de excitación de la muestra.
Como la mayoría de fluoróforos orgánicos presentan su luminiscencia a través de la excitación de transiciones dipolares eléctricas, pérdia de energía por fonones, y re-emisión a través otra transición dipolar, todos fenómenos que ocurren en el orden de los nanosegundos, es posible estudiar su espectro dinámico al excitar con un pulso del láser.
En el caso de las UCNPs, su luminiscencia se da por la dinámica no lineal de la interacción entre sus dopantes lantánidos (Yb$^{+3}$ y Er$^{+3}$), procesos que incluyen la excitación sucesiva sus electrones y por lo tanto ocurren en el orden de los microsegundos.
Por este motivo, es necesario iluminar a la muestra por algunos milisegundos para asegurarse de llegar al estado estacionario del sistema antes de medir su decaimiento.
Esto se hace aprovechando la función de alimentación pulsada (\textit{Quasi Continuous Wave} ó QCW) que ofrece la fuente ITC4020, la cual permite configurar frecuencia de pulsado $\nu$, y ciclo de trabajo $dc$ (\textbf{Fig. \ref{fig:diag_medicion_dinamica}}).

Para hacer una medición de TCSPC es necesario definir la longitud de onda $\lambda$ en la que se detectará la emisión, el intervalo de tiempo $t_{max}$ en el que se van a contar los fotones luego del trigger, y la cantidad de veces $n$ que se va a medir ese intervalo.
Entonces, el protocolo para realizar la medición es:

\begin{enumerate}
     \item \textbf{Iniciar el láser en modo QCW} para que se prenda y se apague con frecuencia $\nu$ y ciclo de trabajo $dc$.
     \item \textbf{Configurar }
\end{enumerate}


\begin{figure}
     \centering
     \includegraphics[width=\textwidth]{diag_medicion_dinamica.png}
     \caption{\textbf{Diagrama de medición dinámica}}
     \label{fig:diag_medicion_dinamica}
\end{figure}


%%%%%%% IMAGENES CRUDAS ACA
%
%\section{Funcionamiento del trigger y simulación de pulsos}
%
%Esta sección se basaría principalmente en contar que teníamos problemas con el trigger y contar cómo nos aseguramos de que la RP triggereara bien a través de la medición de la señal serrucho (bastante pava), y la simulación y adquisición de pulsos con la rp.
%
%imagenes tipo
%
%\begin{figure*}[h]
%    \includegraphics[width=\textwidth]{cap3_tmp2.png}
%\end{figure*}
%
%\begin{figure*}[h]
%    \includegraphics{cap3_tmp1.png}
%\end{figure*}
%
%\section{Caracterización del PMT}
%
%acá pienso poner el análisis que hice para ver que tipo de pulsos había en el PMT, graficos como estos (que hice similares para 5 sampling rate distintos de la RP): \\
%\ref{fig:a} \\
%\ref{fig:b} \\
%\ref{fig:c} \\
%\ref{fig:d} \\
%\ref{fig:e} \\
%\ref{fig:h} \\
%
%todo me parece muy boludo la verdad, lo único interesante es la autocorrelación...
%
%
%\section{Conteo de fotones}
%
%Acá pensaba poner 
%\begin{enumerate}
%    \item lo que daría la distribución de distancia entre fotones si la prob de medir un foton es uniforme
%    \item mostraría lo que da la distribución para distintos métodos de conteo de picos para los datos que medí yo.
%\end{enumerate}
%
%\section{validación patrón}
%
%Acá va:
%
%\begin{enumerate}
%    \item comparación de medición de espectro de excitación y absorción de la rhodamina medido con felix y con nuestro refurbishment
%    \item comparación del tiempo de vida de nuestras mediciones con las de juan u algún otro paper.
%\end{enumerate}
%
%tengo:
%
%\begin{itemize}
%    \item simulación de pulsos con la rp conectada a si misma
%    \item medición de ventanas distintas trigger
%    \item simulación de picos y lo que debería dar conteo correcto
%    \item conteo de picos con distintas técnicas
%    \item imágenes de los picos a varias SR
%    \item altura de picos en func de la altura
%    \item ancho picos en funcion de la altura Y SR
%    \item autocorrelación
%\end{itemize}