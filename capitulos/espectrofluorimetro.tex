
\section{Espectrometría de fluorescencia}

\subsection{Instrumentación: el espectrofluorímetro}

Para caracterizar la respuesta óptica de una sustancia, generalmente se desea registrar tanto el espectro de excitación como el de emisión. 
El instrumento científico por excelencia para realizar estas mediciones es el espectrofluorímetro. 
Fundamentalmente, este instrumento permite realizar mediciones de la intensidad de luz que emite una muestra, haciendo escaneos en longitud de onda de emisión y excitación.
Adicionalmente, algunos pueden realizar mediciones de timepo de vida resueltas en el timpo, escaneos sincrónicos con alguna señal y mediciones de anisotropía en la polarización de materiales luminiscentes.
Estas capacidades son fundamentales para investigaciones en diversas disciplinas científicas, incluyendo química, bioquímica, farmacología, ciencias ambientales, ciencia de materiales y biomedicina.
El objetivo principal de un espectrofluorímetro es obtener los espectros de emisión y de excitación de una muestra.
Para lograr este objetivo, el instrumento debe ser capaz de iluminar la muestra con múltiples longitudes de onda diferentes y registrar su respuesta a cada una de ellas.

La Figura 2.1 muestra un diagrama esquemático de un espectrofluorímetro genérico.
Este instrumento incluye todos los componentes clave para cumplir su función. 
Utiliza una lámpara de espectro amplio que funciona como fuente de luz de alta intensidad para un rango extenso de longitudes de onda. 
Posteriormente la luz es filtrada por un monocromador de excitación que permite seleccionar la longitud de onda con la que se desea iluminar a la muestra.
La luz de excitación seleccionada se enfoca sobre la muestra colocada en la cámara principal, cuya luminiscencia, generalmente con una longitud de onda mayor que la luz de excitación, es filtrada por el monocromador de emisión.
Los monocromadores suelen estar motorizados, lo que permite que el escaneo sea automático. 
La luz restante llega a un detector, usualmente un tubo fotomultiplicador (PMT), un detector muy sensible que convierte fotones en corriente eléctrica.
El espectrofluorímetro emplea diversas técnicas para reducir la luz parásita (longitudes de onda diferentes a la deseada), como el diseño en ángulo de 90° entre los brazos de excitación y emisión, y un compartimento hermético pintado de negro no reflectante.
La señal del PMT es procesada electrónicamente, digitalizada y analizada en una computadora. 
Este sistema también controla los monocromadores y la adquisición de datos, además de permitir al usuario ajustar parámetros y facilitar la visualización y análisis de los datos.
A menudo, se incorporan componentes adicionales en el camino óptico, como obturadores, polarizadores, divisores de haz y otros elementos ópticos, para estudiar diferentes propiedades de la muestra.

\begin{enumerate}
    \item que mide un espectrofluorimetro?
    \item cuales son sus componentes?
    \item que variables puede controlar?
\end{enumerate}

\begin{SCfigure}
     \centering
     \includegraphics[width=0.6\textwidth]{spectrometer_diagram_lako.png}
     \caption{
    \textbf{Diagrama de un espectrofluorímetro genérico}
    \todo{Trabajar en estafigura}
    (\textbf{A}) Diagram of the Horiba PTI QuantaMaster hardware. Red arrows represent motors and limit switch connectors, black is BNC, blue is USB and orange represents a fiber optic. The path that light takes inside the spectrometer is represented in thick blue arrows. 
    (\textbf{B}) and (\textbf{C}) Representation of the old and new instrumental control module respectively.
    (\textbf{D}) Representation of the raw signal measured from the PMT detector.
    (\textbf{E}) Spectrum of the sample constructed from the raw signals measured at each wavelength.
    }
     \label{fig:spec_diagram_lako}
\end{SCfigure}



\subsection{Epectrofluirimetros en argentina y obsolescencia}

Actualmente, el Departamento de Física de la FCEyN-UBA no cuenta con espectrofluorímetros para la caracterización de espectros de excitación y emisión.  
Para realizar este tipo de mediciones, la facultad dispone del laboratorio de fotoquímica del \todo{INQUIMAE}, que cuenta con tres espectrofluorímetros con distintas características, pero que tienen algo en común: ningún equipo tiene menos de 20 años, el más antigüo llegando a los 40 años de uso.
La disparidad en al antigüedad y funcionalidad de los instrumentos es un fenómeno común en laboratorios de investigación en países como Argentina, donde la inversión en ciencia es escasa o poco regular en el tiempo \cite{cioccaRealityScientificResearch2017}. 
Ante esta realidad, los institutos suelen priorizar la adquisición de equipos con nuevas capacidades, en lugar de renovar instrumentos existentes por versiones más modernas.  
Esto es posible gracias a la precisión y robustez de las partes mecánicas de los instrumentos, pero la obsolescencia de los equipos antiguos plantea problemas a largo plazo, especialmente cuando sus plataformas de control quedan desactualizadas.  
Con el tiempo, se vuelve complicado operar estos instrumentos, ya que los mecanismos de extracción de datos, como los disquetes, dejan de estar disponibles en el mercado. 
Aún más crítico es que el funcionamiento del equipo depende de la computadora de control, la cual utiliza placas y puertos que ya no se fabrican ni se consiguen en el mercado actual.  
El problema de la obsolescencia en instrumentos científicos afecta desproporcionadamente a instituciones con bajo presupuesto, ampliando la brecha de accesso a instrumentos de investigación avanzados.

Esto ha llevado a un auge en el desarrollo de instrumentos científicos accesibles y de bajo costo \cite{wenzel_open_2023, arancio_inequalities_2023}, particularmente en áreas como instrumentación, microscopía, espectroscopía y adquisición de datos \cite{jameson_fluorescent_1989, li_optical_2022, hu_fluorescent_2022}.
Las iniciativas de hardware abierto hacen que los diseños y la documentación estén disponibles de forma gratuita para que cualquier persona pueda usarlos, construirlos y modificarlos \cite{powell_democratizing_2012, oellermann_open_2022}.
Por ejemplo, la plataforma Arduino ha proporcionado una plataforma de desarrollo de electrónica económica y fácil de usar basada en un microcontrolador (https://www.arduino.cc/).
El OpenFlexure Microscope es un microscopio de código abierto que cuesta menos de 100 USD construir \cite{collins_robotic_2020}.
Asimismo, recientemente se desarrolló un espectrómetro basado en Raspberry Pi que cuesta menos de 400 EUR \cite{tunens_optical_2024}.
El software y los lenguajes de código abierto, como Python (http://www.python.org), que cuentan con bibliotecas numéricas y de instrumentación como NumPy \cite{harris_array_2020} y PyVISA \cite{grecco_pyvisa_2023}, han desempeñado un papel clave al reducir las barreras de entrada y facilitar la creación rápida de prototipos.
Cabe destacar que han surgido empresas enfocadas en hardware parcialmente o completamente abierto. Por ejemplo, OpenBCI (https://openbci.com/), que ofrece sistemas EEG de bajo costo para interfaces cerebro-computadora, y Opentrons (https://opentrons.com/), que proporciona soluciones de manejo de líquidos para la automatización de laboratorios.

En este capítulo de la tésis se explica detalladamente la plataforma de fuente abierta que desarrollamos para renovar la electrónica y el \textit{software} de control del espectrofluorímetro Horiba PTI QuantaMaster (QM) 400, uno de los espectrofluorímetros antigüos del laboratorio de fotoquímica del INQUIMAE.
Además de su renovación, ampliamos sus capacidades para medir tiempos de vida del orden de los microsegundos, que junto con el agregado de un láser pulsado externo de 980 nm nos permitió conseguir la plataforma ideal para la caracterización óptica de UCNPs tanto estática como dinámica.


\section{Espectrofluorímetro Horiba PTI Quanta Master 400}

\subsection{componentes}

La serie Horiba PTI QM incluye espectrofluorímetros modulares para investigación científica y sistemas optimizados para mediciones de fotoluminiscencia.
Estos espectrofluorímetros se encuentran frecuentemente en los laboratorios de Argentina, por ejemplo, sabemos que hay tres equipos de esta serie en el laboratorio de fotoquímica del INQUIMAE (QM400, QM-4 y RatioMaster), dos en \todo{CIBION} y uno en \todo{CAC-CNEA}, y probablemente haya más de modelos similares en otras instituciones.
Al ser modelos antigüos y descontinuados se pueden encontrar en el mercado por precios que rondan los \$5000 USD, un costo relativamente bajo para un espectrofluorímetro científico.
Los bajos costos se dan por su antigüedad y fin de soporte por parte de la empresa, lo que obliga a los usuarios a resolver ellos mismos los problemas que haya con los equipos.
Por ejemplo, en CIBION, uno de los dos modelos que tienen no está funcionando porque hay problemas con la inicialización de los controladores en la PC.
En este trabajo, reacondicionamos específicamente un espectrofluorímetro QM 400 de más de 30 años de antigüedad,  (diagrama en la \textbf{Fig. \ref{fig:ref-diagram}A} y fotografía en la \textbf{Fig. \ref{fig:hardware}A}), pero dada la similaridad entre los distintos modelos de esta serie, la renovación se pueden aplicar a cualquiera de ellos con leves modificaciones. 

El QM 400 está equipado con una lámpara de xenón de 75 W como fuente de luz, la cual proporciona un amplio espectro de longitudes de onda (desde el infrarrojo cercano, alrededor de 1000 nm, hasta el ultravioleta, alrededor de 300 nm).
Los monocromadores de excitación y emisión contienen redes de difracción rotadas por motores paso a paso de 200 pasos por revolución, con especificaciones de 7 V y 0.7 A por bobina (M1 y M2), lo que permite una resolución en la selección de longitudes de onda de 0.5 nm. 
Ambos incluyen un fin de carrera electromecánico para verificar si se ha alcanzado la longitud de onda máxima.
Los motores paso a paso, junto con sus fines de carrera respectivos, están conectados a un módulo controlador de motores (MDM) mediante conectores propietarios no documentados. 
Los fotones son detectados por un tubo fotomultiplicador (PMT, modelo PTI 810), cuya caracterización se detalla en la sección \ref{sec:caracterizacion_pmt}.
Éste está conectado al MDM a través de un cable BNC y polarizado con 1000 V desde una fuente de alimentación externa proporcionada también por el MDM. 
Esto genera pulsos negativos de alrededor de 170 ns con una terminación de 50 Ohm y un voltaje de −3.5 V (\textbf{Fig. \ref{fig:ref-diagram}D}). 
Finalmente, el MDM está conectado mediante un cable plano a una tarjeta de interfaz ISA en una PC con sistema operativo Windows 95 y el programa FelixGX, un \textit{software} propietario de adquisición y control instalado por Horiba (\textbf{Fig. \ref{fig:ref-diagram}B}).
FelixGX permite medir espectros de emisión y excitación (\textbf{Fig. \ref{fig:ref-diagram}E}), además de brindar herramientas de análisis rápido de los datos y controlar diferentes periféricos.

La antigüedad de la PC y electrónica de control hace que el proceso de adquisición de datos sea tedioso, y más aún para la caracterización de UCNPs.
Para hacer una medición el usuario debe colocar la muestra en la cámara y luego configurar en FelixGX un barrido de la longitud de onda de excitación o emisión.
En el caso de medir \textit{upconversion} se debe agregar un láser controlado externamente por una fuente de corriente (\textbf{Fig. \ref{fig:ref-diagram}A}) en la que se debe configurar por separado los parámetros de excitación, como la potencia.
Una plataforma de caracterización óptica completa de UCNPs debería ser capaz de medir espectros de excitación a 980 nm con distintas densidades de potencia, y tiempos de vida (también a distintas potencias) del orden de los microsegundos.
Además, como las mediciones de espectro y tiempo de vida son de larga duración (en especial a bajas potencias), resulta ideal que la plataforma permita configurar múltiples mediciones sucesivas sin la necesidad de una configuración manual por el usuario.
En las siguientes secciones, explicaremos los cambios de \textit{hardware} y \textit{software} que realizamos en el espectrofluorímetro para que sea una plataforma ideal para medir \textit{upconversion}.


%\todo{ \\
%serie horiba quantamaster. frecuencia de aparición en exactas y arg en general, baratos, etc. quizás mencionar lo de stefani \\
% \\
%componentes de funcionamiento: conectores originales, lampara, monocromadores, chamber, pmt, especificaciones \\
% \\
%software de control: felix gx, capacidades fundamentales y deficiencias \\
% \\
%que falta para caracterizar ucnps? time-consuming experiments, operación, medición de tiempos de vida, etc \\
%}

\begin{figure}[btp]
     \centering
     \includegraphics[width=\textwidth]{spec_diagram.png}
     \caption{
     \textbf{Representación esquemática del espectrofluorímetro}
     (\textbf{A}) Diagrama del hardware del Horiba PTI QM 400. Las flechas rojas representan los conectores de motores y fines de carrera, las negras corresponden a BNC, las azules a USB y las naranjas representan fibra óptica. La trayectoria de la luz dentro del espectrómetro está indicada con flechas azules gruesas.
     (\textbf{B}) y (\textbf{C}) Representación del módulo de control instrumental antiguo y nuevo, respectivamente.
     (\textbf{D}) Representación de la señal cruda medida por el detector PMT.
     (\textbf{E}) Espectro de la muestra construido a partir del conteo de picos en las señales crudas medidas para cada longitud de onda.
    }
     \label{fig:ref-diagram}
\end{figure}

\begin{figure}[h]
     \centering
     \includegraphics[width=0.9\textwidth]{hardware.png}
     \caption{\textbf{Horiba PTI QuantaMaster 400 picture}. \todo{maybe pasar a apéndice} (\textbf{A}) Picture of the whole spectrometer. Circled in red the monochromators' motors and limit switches. (\textbf{B}) Stepper motors pin diagram. The only used pins for the refurbished version are 1 and 7, and 3 and 5, which correspond to each motor winding respectively. (\textbf{C}) Limit switches pin diagram.}
     \label{fig:hardware}
\end{figure}



\section{Renovación y ampliación de Horiba PTI Quanta Master 400 \todo{mencionar a juan}}

\subsection{Hardware para renovación}

Luego de hacer una inspección de todas las partes, decidimos conservar los componentes ópticos, la motorización, el PMT, la fuente de alta tensión y el chasis, ya que son robustos y funcionales.  
En contraste, la electrónica de control y detección resultó ser voluminosa, de código cerrado y obsoleta, por lo que optamos por reemplazarla con alternativas modernas: una microCPU con FPGA integrada Red Pitaya (RP) STEM LAB 125-14.
La RP cuenta con cuatro entradas y salidas analógicas que emiten y procesan señales en las radiofrecuencias, y un conjunto de pines digitales que permiten controlar circuitos integrados fácilmente (\textbf{\ref{fig:connection_diagram}}).
Esta placa junto con dos circuitos integrados DRV8825 que simplifican el control de los motores por paso cumplen la tarea de controlar a los monocromadores.
Este cambio en la electrónica de control nos permitió reemplazar el voluminoso módulo MDM ($\sim$10 cm $\times$ 30 cm $\times$ 30 cm) por dos controladores DRV8825 soldados a una placa PCB mucho más pequeña (10 cm$\times$ 10 cm $\times$ 2 cm)(\textbf{Fig. \ref{fig:placa}}).
Para facilitar la conexión de los componentes del espectrofluorímetro a la RP también agregamos a la placa un puerto IDC que permite conectar los pines digitales, y conectores a los motores a través de fichas adaptadas a medida.
El PMT se conecta a través de un cable BNC-SMA a uno de los canales analógicos de radiofrecuencias de la RP, luego son digitalizados por su conversor analógico digital (ADC)(\textbf{Fig. \ref{fig:ref-diagram}D}) y luego contados por software (\textbf{Fig. \ref{fig:ref-diagram}E}).
El ADC de 14 bits de la RP se configura con una frecuencia de muestreo de 32.25 MHz de forma tal de satisfacer el criterio de Nyquist.
La API de la RP permite configurar dos métodos distintos para comenzar una adquisición de datos.
Una opción es llamar a una función que comienza la adquisición de inmediato.
Alternativamente, se puede configurar una de las entradas analógicas o digitales como \textit{trigger} para comenzar una medición.
Nosotros usamos un método para medir espectros estáticos y otro para medir tiempos de vida.
Asimismo, la RP tiene dos mecanismos distintos para escribir los datos en la memoria al hacer una adquisición: un método por defecto de escritura a un espacio de memoria de 2$^{14}$ enteros de 16 bits, y un método de adquisición de memoria profunda (DMA) que permite guardar hasta 2 MB \cite{DMA_rp}.
Con esta capacidad para almacenar datos, a 32.25 MHz la ventana temporal de pulsos más grande que se puede obtener es de $\sim 0.5$ ms y $\sim 8$ ms respectivamente.
Aunque resulta beneficioso el método DMA y se podría implementar en esta renovación, nosotros utilizamos el método por defecto por la simplicidad que significó durante el desarrollo.
Para contrarrestar la corta duración de la ventana de adquisición, el software de control que desarrollamos permite agregar una demora para obtener ventanas de medición más grandes (ver sección \ref{sec:proceso_dinamico}).
El apéndice \ref{apendice:instrucciones_armado} explica detalladamente cómo reproducir las conexiones.

\begin{figure}
     \centering
     \includegraphics[width=0.8\textwidth]{connection_diagram.png}
     \caption{
    \textbf{Conexiones necesarias para la renovación y ampliación del QM400.}
    }
     \label{fig:connection_diagram}
\end{figure}


%\todo{Qué reemplazamos y con qué nos quedasmos, especificaciones finales}

\subsection{Hardware ampliación}

El espectrofluorímetro original QM 400 disponible en el laboratorio no era adecuado para estudiar \textit{upconversion}, ya que no contaba con una fuente de luz en el infrarrojo (IR). 
Tampoco era posible realizar mediciones de tiempos de vida de la luminiscencia debido a la falta de excitación pulsada y detección dependiente del tiempo. 
Luego de aplicar renovación mencionada anteriormente, incorporamos estas funcionalidades al equipo y al \textit{software} de forma independiente.

Para ello, añadimos una fuente de luz IR externa modulable al sistema. 
En nuestro caso, utilizamos un controlador de diodo láser y temperatura (TEC) de banco THORLABS ITC4020, controlado por la RP, para operar un diodo láser BL976-SAG300 de 976 nm y 300 mW. 
La salida del diodo láser se conecta mediante una fibra óptica a la entrada de fuente externa del QM 400 (\textbf{Fig. \ref{fig:ref-diagram}A}). 
El ITC4020 permite configurar la frecuencia de pulsado y el ciclo de trabajo, además de proporcionar una señal TTL que está en 5 V cuando el láser está prendido y 0 V cuando está apagado.
Esta señal se conecta a otra de las entradas analógicas de la RP, que luego la utiliza como \textit{trigger} para sincronizar la finalización de la excitación del láser, con la medición de los pulsos eléctricos de los fotones, para luego realizar histogramas y así medir los tiempos de vida, proceso que se explica en detalle en la sección \ref{sec:proceso_dinamico}.  

%\todo{ \\
%Excitación con laser pulsado. Control de potencia y duty cycle. funcionamiento del trigger. el resto ya lo puede hacer la RP \\
%}

\begin{figure}[h]
     \centering
     \includegraphics[width=0.9\textwidth]{placa.jpg}
     \caption{\textbf{Horiba PTI QuantaMaster 400 picture}. \todo{maybe pasar a apéndice} (\textbf{A}) Picture of the whole spectrometer. Circled in red the monochromators' motors and limit switches. (\textbf{B}) Stepper motors pin diagram. The only used pins for the refurbished version are 1 and 7, and 3 and 5, which correspond to each motor winding respectively. (\textbf{C}) Limit switches pin diagram.}
     \label{fig:placa}
\end{figure}


\subsection{Caracterización del PMT con la Red Pitaya} \label{sec:caracterizacion_pmt}

 
Los tubos fotomultiplicadores (PMTs) son detectores ampliamente utilizados en fluorómetros y cumplen un rol central en el funcionamiento del instrumento. 
Un PMT está compuesto por un fotocátodo y una serie de dínodos, que actúan como etapas de amplificación. 
Los fotones incidentes provocan la emisión de electrones desde el fotocátodo, y estos son amplificados sucesivamente por los dínodos. 
Al final del proceso, que tiene una duración típica de 40 a 50 nanosegundos, el pulso llega a un ánodo, desde donde se lee la señal. 
Este detector puede operar en dos modos: analógico o conteo de fotones. 
En el modo analógico, la corriente que fluye por el ánodo debe ser proporcional a la intensidad de luz que incide sobre el fotocátodo.
En el modo de conteo, el PMT registra pulsos eléctricos de corta duración cada vez que se detecta un fotón. 
Resulta crucial que el detector no se encuentre dañado, cosa que usualmente sucede por la exposición a luz muy intensa. 
Esto puede generar una corriente de oscuridad alta, pulsos espurios en la señal que sesgan el resultado de la medición \cite{lakowicz_principles_2006}.
En nuestro instrumento renovado, la señal del PMT se adquiere utilizando una Red Pitaya (RP). 
Este dispositivo permite una frecuencia máxima de adquisición de 125 MHz, aunque también es posible trabajar a frecuencias menores que sean divisiones por 2 de ese valor. 
Para realizar un conteo preciso de los fotones que inciden en el fotocátodo, caracterizamos la señal del PMT del QM 400 utilizando la RP en las mismas condiciones operativas del fluorímetro modificado.
Como resultado, determinamos el ancho medio de los pulsos y definimos el algoritmo de detección de pulsos más adecuado para optimizar el conteo. 

Como comentamos en la sección anterior, la RP puede medir una ventana temporal de hasta $2^{14}$ muestras.
Una señal típica del PMT al ser iluminado de forma constante, medida con una ventana de la RP a 31.25 MHz,  se puede ver en la figura (\textbf{\ref{fig:pmt_signal}A}).
La señal muestra un voltaje de base que ronda los 3.5 V, y múltiples picos negativos que representan la llegada de un fotón al fotocátodo.
Si bien la mayoría de pulsos suelen estar entre 1 y 0 V, hay otros que no llegan a saturar el detector y tienen menor altura.
Los pulsos que saturan tienen un tiempo rápido de subida, de aproximadamente $\sim 16$ ns, luego son constantes y después bajan rápidamente.  
Se puede nota una ligera asimetría, sus colas presentan dos rebotes alrededor de 2 y 3 V (\textbf{Fig. \ref{fig:pmt_signal}B}).
Por otro lado, los picos que no saturan presentan oscilaciones de alta frecuencia en el máximo.
Si se compara una adquisición a distintas frecuencias de muestreo, se puede ver que estos efectos se ven atenuados al medir a menores frecuencias de muestreo (\textbf{Fig. \ref{fig:pmt_signal}C}). 

Seleccionando la frecuencia de 31.25 MHz se puede ver el histograma  del voltaje de los puntos de la señal (\textbf{Fig. \ref{fig:histograma_puntos}}).
Para eliminar la base de la señal se tomaron sólo los puntos con voltajes menores a 3.4 V.
Se pueden ver dos picos en el histograma, uno para voltajes menores a 1 V, que corresponde a los pulsos que saturan al PMT, y otro para mayores a 3 V, que se pueden adjudicar al ruido y a los rebotes de la señal en las colas de los pulsos.
El eje vertical en escala logarítmica permite diferenciar un salto alrededor de 1.4 V, éste corresponde a los pulsos que no saturan y cuyas oscilaciones de alta frecuencia se ven atenuadas a 31.25 MHz.
Viendo el histograma de la altura de los pulsos, tomamos como criterio considerar un pulso de la señal como un fotón que llegó al detector a todos ellos cuyo pico sea menor a 1 V.

Por último, para estimar el ancho promedio de los pulsos calculamos la autocorrelación a partir de la medición de múltiples ventanas de la señal del PMT bajo iluminación constante.
La relación entre la desviación estandar $\sigma$ de una distribución gaussiana, y la desviación estandar $\sigma_c$ su autocorrelación está dada por $\sigma = \sigma_c/\sqrt{2}$.
Aproximando a los pulsos que llegan al detector por gaussianos, y usando que el ancho total a mitad de altura (FWHM) de una gaussiana es FWHM $= 2 \sqrt{2 \ln{2}}\sigma$, podemos obtener el FWHM de los pulsos con la fórmula FWHM $= 2 \sqrt{2 \ln{2}}\sigma_c / \sqrt{2}$
Por lo tanto, ajustamos el resultado de las autocorrelaciones por una gaussiana y realizamos un histograma de las desviaciones estandar de los pulsos $\sigma = \sigma_c / \sqrt{2}$ obtenidas (\textbf{Fig. \ref{fig:ancho_pulsos}}).
Como resultado, el ancho temporal de los pulsos a mitad de altura es de $(\Delta t = 113 \pm 2)$ ns, donde definimos el error como dos desviaciones estándar de la media.
Esto nos permite definir una tasa máxima $R$ de fotones por segundo que podemos detectar con el PMT.
Bajo iluminación continua, la probabilidad de que un fotón llegue al detector en un tiempo $t$ al medir en un intervalo $T$ es uniforme.
Entonces, la probabilidad $P$ de que otro fotón se solape debe ser $P = R \times T$, por lo que si queremos que la probabilidad máxima sea del 90\%, podremos adquirir a una tasa de fotones por segundo de $R = P/T = 0.1/113 \times 10^9$ Hz$ = 8.9 \times 10^5$ Hz.



\begin{figure}
    \centering
    \includegraphics[width=\textwidth]{sig_pmt.png}
    \caption{\textbf{Señal del PMT} tomada con la RP a 32.25 MHz.}
    \label{fig:pmt_signal}
\end{figure}

\begin{SCfigure}
    \centering
    \includegraphics[width=0.5\textwidth]{histograma_puntos.png}
    \caption{\textbf{Señal del PMT} tomada con la RP a 32.25 MHz.}
    \label{fig:histograma_puntos}
\end{SCfigure}


\begin{figure}
    \includegraphics[width=\textwidth]{ancho_pulsos.png}
    \caption{autocorrelacion de los picos. Tmb tengo un fit gausiano para ver el ancho.}
    \label{fig:ancho_pulsos}
\end{figure}