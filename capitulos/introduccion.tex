
\begin{enumerate}
    \item que es la luminiscencia? algunas aplicaciones ej. microscopía, plim y flim
    \begin{itemize}
        \item In 1565, a Spanish physician and botanist, Nicolas Monardes (Figure 2), reported the peculiar blue color (under certain  conditions of observation) from an infusion of a wood from Mexico and used to treat kidney and urinary diseases (Figure 3). 912 This wood (later called Lignum nephriticum), whose peculiar color effect and diuretic properties were already known to the Aztecs, was a scarce and expensive medicine. Therefore, it was of interest to detect counterfeited wood. Monardes wrote on this respect, 12 Make sure that the wood renders water bluish, otherwise it is a falsification. Indeed, they now bring another kind of wood that renders the water yellow, but it is not good, only the kind that renders the water bluish is genuine. (in Spanish in the original). This method for the detection of a counterfeited object can be considered as the first application of the phenomenon that would be later called fluorescence. Extracts of the wood were further investigated by Boyle, Newton, and others, 6 but the phenomen- on was not understood at the time.
        \item n 1845, the polymath Sir John Herschel, son of the famous astronomer and the originator of the word photography (where light is again involved), prepared an acid solution of quinine sulfate and stated,18 Though perfectly transparent and colorless when held between the eye and the light, it yet exhibits in certain aspects, and under certain incidences of the light, an extremely vivid and beautiful celestial blue color. As the color was always superficial, he believed it to be a hitherto unidentified phenomenon, “a case of superficial colour presented by a homogeneous liquid, internally colourless”. 18 Herschel called this phenomenon epipolic dispersion, from the Greek: επιπολη = surface. In fact, the solutions observed by Herschel were very concentrated so that the majority of the incident light was absorbed near the surface and all the blue fluorescence originated from there. Herschel used a prism to show that the epipolic dispersion could be observed only upon illumination by the blue end of the spectrum and not the red end. The crude spectral analysis of the emitted light with the prism revealed blue, green, and a small quantity of yellow light, but Herschel did not realize that the superficial light was of longer wavelength than the incident light.
        \item One of Stokes’s experiments that is spectacular and remarkable by its simplicity deserves attention. Stokes formed the solar spectrum by means of a prism. When he moved a test tube filled with a solution of quinine through the visible part of the spectrum, nothing happened: the solution remained transparent. 20 But beyond the violet portion of the spectrum, that is, in the invisible zone corresponding to ultraviolet radiation, the solution glowed with a blue light (Figure 7). Stokes wrote, 4 It was certainly a curious sight to see the tube instantaneously light up when plunged into the invisible rays: it was literally darkness visible. Altogether the phenomenon had something of an unearthly appearance. From his experiments with a wide range of substances, Stokes concluded that the dispersed light was always of longer wave- lengths than the incident light. Later this statement became the Stokes law. Stokes also noted that the dispersion of light took place in all directions, hence, the fluid behaved as if it were self-luminous. In his paper, Stokes called the observed phenomenon true internal dispersion or dispersive reflection but in a footnote, 4 he wrote, I confess I do not like this term. I am almost inclined to coin a word, and call the appearance fluorescence, from fluorspar, as the analogous term opalescence is derived from the name of a mineral. In his second paper, 21 Stokes definitely resolved to use the word fluorescence.
        \item Luminescence is an emission of ultraviolet, visible or infrared photons from an electronically excited species. The word luminescence, which comes from the Latin (lumen ¼ light) was first introduced as luminescenz by the physicist and science historian Eilhardt Wiedemann in 1888, to describe ‘all those phenomena of light which are not solely conditioned by the rise in temperature’, as opposed to incan- descence. Luminescence is cold light whereas incandescence is hot light. The various types of luminescence are classified according to the mode of excitation (see Table 1.1).
        \item The success of fluorescence as an investigative tool in studying the structure and dynamics of matter or living systems arises from the high sensitivity of fluoro- metric techniques, the specificity of fluorescence characteristics due to the micro- environment of the emitting molecule, and the ability of the latter to provide spatial and temporal information. Figure 1.3 shows the physical and chemical parameters that characterize the microenvironment and can thus affect the fluorescence char- acteristics of a molecule.
        \item As a consequence of the strong influence of the surrounding medium on fluores- cence emission, fluorescent molecules are currently used as probes for the investi- gation of physicochemical, biochemical and biological systems. A large part of this book is devoted to the use of so-called fluorescent probes.
        \item An electronic transition consists of the promotion of an electron from an orbital of a molecule in the ground state to an unoccupied orbital by absorption of a photon. The molecule is then said to be in an excited state. Let us recall first the various types of molecular orbitals.
        \item Additionally, fluorescence is used for cell identification and sorting in flow cytometry, and in cellular imaging to reveal the localization and movement of intracellular substances by means of fluores- cence microscopy.
        \item It is interesting to notice that the first known fluoro- phore, quinine, was responsible for stimulating the devel- opment of the first spectrofluorometers, which appeared in the 1950s. During World War II, the Department of De- fense was interested in monitoring antimalaria drugs, in- cluding quinine. This early drug assay resulted in a subsequent program at the National Institutes of Health to develop the first practical spectrofluorometer.*
    \end{itemize}
    \item hamiltoniano de un átomo de muchos electrones? - si hago esto, el foco debería ser \\
    por qué las cosas emiten o no emiten?
    \item diagrama de jablonski. Cómo se convierte la excitación del electrón en luz? \\
    de qué otras formas puede convertirse la energía del electrón?
    decaimientos radiativos y no radiativos. \\
    ver seccion jablonski del lakowicz
    \item fluorescencia y fosforescencia diferencias -> transiciones prohibidas y tiempos de vida
    \item corrimiento stokes y anti stokes -> fenómenos de muchos fotones
    \item caracterización de luminiscencia estática
    \item caracterización de luminiscencia dinámica -> TCSPC
    \item Nanopartículas de upconversion, composición y aplicaciones
    \item fotofísica de los lantánidos
    \item mecanismos principales de upconversión
    \item dependencia con la potencia de excitación
    \item modelos de la dinámica de las nanopartículas
    \item que hace falta para caracterizar las nanopartículas?
\end{enumerate}


\newpage


\todo{agregar fenómenos de fosforescencia más antiguos}
En 1565, el médico y botanista Nicolás Monardes reportó el peculiar color azul que tomaba una infusión de madera mexicana usada para tratar enfermedades de riñón y urinarias. Este efecto ya era conocido por los Aztecas, que lo utilizaban para asegurarse que la valiosa madera no fuera falsificada. Monardes escribe en su libro \cite{[MONARDES]}

\begin{quote}
    \textit{Asegúrate de que la madera torne el agua azulada, de lo contrario, es una falsificación. De hecho, ahora traen otro tipo de madera que torna el agua amarilla, pero no sirve; solo el tipo que torna el agua azulada es genuina.}
\end{quote}  

\noindent Años más tarde, en 1845, el matemático Sir John Herschel describió el efecto similar que producía una solución transparente de quinina, una sustancia presente en el agua tónica, que reflejaba \flqq un color azul celestial hermoso y extremadamente vívido\frqq.
Herschel usó un prisma para comprobar que la dispersión causada por la quinina sólo se observaba al iluminar la solución con la parte azul del espectro.
El mismo análisis para la luz emitida reveló luz azul, verde, y una pequeña cantidad de amarillo.
En esa misma época, el físico Sir George Gabriel Stokes publicó \textit{On the Refrangibility of Light}, un trabajo explicando experimentos con múltiples sustancias que exhibían este tipo de comportamientos, entre ellas incluida la quinina.
Uno de sus experimentos más importantes consistía en formar el espectro solar a partir de un prisma, para luego mover un tubo de ensayo con la solución de quinina a través de sus colores.
La solución permanecía transparente al ser iluminada por la parte visible del espectro, pero al llegar a la zona ultravioleta (invisible al ojo humano), la muestra se iluminó con luz azul brillante.
Además de concluir que la luz siempre se dispersaba con longitudes de onda mayores a las de incidencia, afirmación que luego se conocería como corrimiento de Stokes, llamó a este fenómeno \textit{fluorescencia} \cite{[VALEUR HISTORY]}.

En 1888, el físico Eilhardt Wiedmann introdujo el término luminiscencia para referirse a los fenómenos lumínicos que no están determinados por un aumento en la temperatura de los materiales.
El desarrollo de la mecánica cuántica durante el siglo 20 dió a conocer el fenómeno detrás de la emisión de luz sin aumento de temperatura: las transición de los electrones entre los distintos niveles de energía de un átomo. 


Entre ellos está la fosforescencia \todo{introducir ejemplo que mencioné antes} y de fluorescencia como el de la quinina descripto por Stokes.
Actualmente, tanto la fluorescencia como la fosforescencia tienen aplicaciones en múltiples areas distintas del conocimiento y la tecnología. 
Es particularmente destacable su éxito como herramienta para estudiar la estructura y dinámica de la materia viva, gracias la sensibilidad al micro-entorno de las moléculas fluorescentes, lo que resulta en una alta resolución espacial y temporal. 
Por ejemplo, la microscopía de fluorescencia consiste en iluminar la muestra con una longitud de onda y detectar su fluorescencia en otra, permitiendo filtrar el fondo de la imagen \cite{VALEUR}.
Técnicas dinámicas como la microscopía de imágenes de tiempo de vida de fluorescencia (FLIM) o fosforescencia (PLIM) dan lugar a conocer el entorno químico en el que se encuentran distintas proteinas \todo{[CITA]}.




\section{Espectrofotometría estática}
\section{Espectrofotometría dinámica} \label{sec:intro_tcspc}
\section{Nanopartículas de conversión ascendente} \label{sec:intro_ucnp}