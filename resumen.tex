
La fotoluminiscencia es la emisión de luz de la materia debida a una absorción previa de fotones. 
Esta propiedad es frecuentemente utilizada en microscopía con aplicaciones bioquímicas y biológicas, generalmente a través de agentes fotoluminiscentes como tintas orgánicas, proteínas fluorescentes y puntos cuánticos semiconductores. 
La mayoría de estos agentes exhiben corrimiento de Stokes, conversión de fotones de alta energía (UV-vis) a baja energía (rojo-NIR).
En este trabajo se estudian las nanopartículas de \textit{upconversion} (UCNP), redes cristalinas dopadas con lantánidos que interactúan entre sí para absorber fotones en el NIR y re-emitirlos en el UV-VIS.

En particular, se desarrolló tanto el \textit{hardware} como el \textit{software} de una plataforma capaz de la caracterización óptica de estas nanopartículas, tomando como base un espectrofluorímetro comercial obsoleto Horiba PTI Quanta Master 400.
Se renovó el instrumento reemplazando la PC y electrónica de control por una CPU y FPGA Red Pitaya.
Se desarrollaron dos paquetes de Python, \textit{redpipy} y \textit{refurbishedPTI}, que perimten controlar los monocromadores y leer la señal del PMT del fluorímetro.
Ambos paquetes se desarrollaron de forma tal que puedan ser reutilizados: \textit{redpipy} consiste en una interfaz de aplicación del programador (API) para controlar la Red Pitaya, y \textit{refurbishedPTI}, que implementa el control del fluorímetro, no es particular de este instrumento, sino que se puede utilizar para controlar cualquier fluorímetro con monocromadores y PMT.
Para asegurarse de que el fluorímetro renovado implemente las funcionalidades correctamente se caracterizó en profundidad el sistema de detección de fotones.
Además, se expandieron las capacidades del fluorímetro original agregando un láser externo pulsado, con una fuente de potencia variable, que permite realizar mediciones de tiempo de vida aplicando la técnica \textit{Time-Correlated Single Photon Counting} (TCSPC).

Por último, se utilizó esta plataforma para caracterizar un lote de UCNP compuestas por una matriz cristalina de fluoruro de ítrio NaYF$_4$ y dopada con iones lantánidos erbio Er$^{3+}$ e iterbio Yb$^{3+}$.
Para eso, se midió el espectro de emisión estacionario al excitar de forma continua con un láser de 976 nm en un rango de densidades de potencia de excitación desde 16 mW cm$^{-1}$ hasta 80 mW cm$^-1$.
Asimismo, se midieron los tiempos de vida para los picos de emisión más intensos, y se comparó el resultado a dos potencias de excitación distintas.
Se observó un comportamiento lineal en función de la potencia para los espectros estacionarios, y no se observaron diferencias significativas entre los tiempos de vida a distintas potencias.

\textbf{Palabras clave:} \textit{upconversion}, luminiscencia, espectrofluorimetría, instrumentación y control.




